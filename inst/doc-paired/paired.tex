% Created 2025-05-13 ti 18:25
% Intended LaTeX compiler: pdflatex
\documentclass[12pt]{article}

%%%% settings when exporting code %%%% 

\usepackage{listings}
\lstdefinestyle{code-small}{
backgroundcolor=\color{white}, % background color for the code block
basicstyle=\ttfamily\small, % font used to display the code
commentstyle=\color[rgb]{0.5,0,0.5}, % color used to display comments in the code
keywordstyle=\color{black}, % color used to highlight certain words in the code
numberstyle=\ttfamily\tiny\color{gray}, % color used to display the line numbers
rulecolor=\color{black}, % color of the frame
stringstyle=\color[rgb]{0,.5,0},  % color used to display strings in the code
breakatwhitespace=false, % sets if automatic breaks should only happen at whitespace
breaklines=true, % sets automatic line breaking
columns=fullflexible,
frame=single, % adds a frame around the code (non,leftline,topline,bottomline,lines,single,shadowbox)
keepspaces=true, % % keeps spaces in text, useful for keeping indentation of code
literate={~}{$\sim$}{1}, % symbol properly display via latex
numbers=none, % where to put the line-numbers; possible values are (none, left, right)
numbersep=10pt, % how far the line-numbers are from the code
showspaces=false,
showstringspaces=false,
stepnumber=1, % the step between two line-numbers. If it's 1, each line will be numbered
tabsize=1,
xleftmargin=0cm,
emph={anova,apply,class,coef,colnames,colNames,colSums,dim,dcast,for,ggplot,head,if,ifelse,is.na,lapply,list.files,library,logLik,melt,plot,require,rowSums,sapply,setcolorder,setkey,str,summary,tapply},
aboveskip = \medskipamount, % define the space above displayed listings.
belowskip = \medskipamount, % define the space above displayed listings.
lineskip = 0pt} % specifies additional space between lines in listings
\lstset{style=code-small}
%%%% packages %%%%%

\usepackage[utf8]{inputenc}
\usepackage[T1]{fontenc}
\usepackage{lmodern}
\usepackage{textcomp}
\usepackage{color}
\usepackage{graphicx}
\usepackage{grffile}
\usepackage{wrapfig}
\usepackage{rotating}
\usepackage{longtable}
\usepackage{multirow}
\usepackage{multicol}
\usepackage{changes}
\usepackage{pdflscape}
\usepackage{geometry}
\usepackage[normalem]{ulem}
\usepackage{amssymb}
\usepackage{amsmath}
\usepackage{amsfonts}
\usepackage{dsfont}
\usepackage{array}
\usepackage{ifthen}
\usepackage{hyperref}
\usepackage{natbib}
\RequirePackage{setspace} % to modify the space between lines - incompatible with footnote in beamer
\renewcommand{\baselinestretch}{1.1}
\geometry{a4paper, left=10mm, right=10mm, top=10mm}
\usepackage{titlesec}
\usepackage{etoolbox}

\makeatletter
\patchcmd{\ttlh@hang}{\parindent\z@}{\parindent\z@\leavevmode}{}{}
\patchcmd{\ttlh@hang}{\noindent}{}{}{}
\makeatother
\RequirePackage{colortbl} % arrayrulecolor to mix colors
\definecolor{myorange}{rgb}{1,0.2,0}
\definecolor{mypurple}{rgb}{0.7,0,8}
\definecolor{mycyan}{rgb}{0,0.6,0.6}
\newcommand{\lightblue}{blue!50!white}
\newcommand{\darkblue}{blue!80!black}
\newcommand{\darkgreen}{green!50!black}
\newcommand{\darkred}{red!50!black}
\definecolor{gray}{gray}{0.5}
\hypersetup{
citecolor=[rgb]{0,0.5,0},
urlcolor=[rgb]{0,0,0.5},
linkcolor=[rgb]{0,0,0.5},
}
\newenvironment{note}{\small \color{gray}\fontfamily{lmtt}\selectfont}{\par}
\newenvironment{activity}{\color{orange}\fontfamily{qzc}\selectfont}{\par}
\RequirePackage{pifont}
\RequirePackage{relsize}
\newcommand{\Cross}{{\raisebox{-0.5ex}%
{\relsize{1.5}\ding{56}}}\hspace{1pt} }
\newcommand{\Valid}{{\raisebox{-0.5ex}%
{\relsize{1.5}\ding{52}}}\hspace{1pt} }
\newcommand{\CrossR}{ \textcolor{red}{\Cross} }
\newcommand{\ValidV}{ \textcolor{green}{\Valid} }
\usepackage{stackengine}
\usepackage{scalerel}
\newcommand\Warning[1][3ex]{%
\renewcommand\stacktype{L}%
\scaleto{\stackon[1.3pt]{\color{red}$\triangle$}{\tiny\bfseries !}}{#1}%
\xspace
}
\definecolor{grayR}{HTML}{8A8990}
\definecolor{grayL}{HTML}{C4C7C9}
\definecolor{blueM}{HTML}{1F63B5}
\newcommand{\Rlogo}[1][0.07]{
\begin{tikzpicture}[scale=#1]
\shade [right color=grayR,left color=grayL,shading angle=60]
(-3.55,0.3) .. controls (-3.55,1.75)
and (-1.9,2.7) .. (0,2.7) .. controls (2.05,2.7)
and (3.5,1.6) .. (3.5,0.3) .. controls (3.5,-1.2)
and (1.55,-2) .. (0,-2) .. controls (-2.3,-2)
and (-3.55,-0.75) .. cycle;

\fill[white]
(-2.15,0.2) .. controls (-2.15,1.2)
and (-0.7,1.8) .. (0.5,1.8) .. controls (2.2,1.8)
and (3.1,1.2) .. (3.1,0.2) .. controls (3.1,-0.75)
and (2.4,-1.45) .. (0.5,-1.45) .. controls (-1.1,-1.45)
and (-2.15,-0.7) .. cycle;

\fill[blueM]
(1.75,1.25) -- (-0.65,1.25) -- (-0.65,-2.75) -- (0.55,-2.75) -- (0.55,-1.15) --
(0.95,-1.15)  .. controls (1.15,-1.15)
and (1.5,-1.9) .. (1.9,-2.75) -- (3.25,-2.75)  .. controls (2.2,-1)
and (2.5,-1.2) .. (1.8,-0.95) .. controls (2.6,-0.9)
and (2.85,-0.35) .. (2.85,0.2) .. controls (2.85,0.7)
and (2.5,1.2) .. cycle;

\fill[white]  (1.4,0.4) -- (0.55,0.4) -- (0.55,-0.3) -- (1.4,-0.3).. controls (1.75,-0.3)
and (1.75,0.4) .. cycle;

\end{tikzpicture}
}
\RequirePackage{fancyvrb}
\DefineVerbatimEnvironment{verbatim}{Verbatim}{fontsize=\small,formatcom = {\color[rgb]{0.5,0,0}}}
\RequirePackage{enumitem} % better than enumerate
\RequirePackage{epstopdf} % to be able to convert .eps to .pdf image files
\RequirePackage{capt-of} %
\RequirePackage{caption} % newlines in graphics
\RequirePackage{tikz-cd} % graph
\RequirePackage{booktabs} % for nice lines in table (e.g. toprule, bottomrule, midrule, cmidrule)
\RequirePackage{amsmath}
\RequirePackage{algorithm}
\RequirePackage[noend]{algpseudocode}
\RequirePackage{dsfont}
\RequirePackage{amsmath,stmaryrd,graphicx}
\RequirePackage{prodint} % product integral symbol (\PRODI)
\usepackage{ifthen}
\usepackage{xifthen}
\usepackage{xargs}
\usepackage{xspace}
\newcommand\defOperator[7]{%
\ifthenelse{\isempty{#2}}{
\ifthenelse{\isempty{#1}}{#7{#3}#4}{#7{#3}#4 \left#5 #1 \right#6}
}{
\ifthenelse{\isempty{#1}}{#7{#3}#4_{#2}}{#7{#3}#4_{#1}\left#5 #2 \right#6}
}
}
\newcommand\defUOperator[5]{%
\ifthenelse{\isempty{#1}}{
#5\left#3 #2 \right#4
}{
\ifthenelse{\isempty{#2}}{\underset{#1}{\operatornamewithlimits{#5}}}{
\underset{#1}{\operatornamewithlimits{#5}}\left#3 #2 \right#4}
}
}
\newcommand{\defBoldVar}[2]{
\ifthenelse{\equal{#2}{T}}{\boldsymbol{#1}}{\mathbf{#1}}
}
\newcommandx\Esp[2][1=,2=]{\defOperator{#1}{#2}{E}{}{\lbrack}{\rbrack}{\mathbb}}
\newcommandx\Prob[2][1=,2=]{\defOperator{#1}{#2}{P}{}{\lbrack}{\rbrack}{\mathbb}}
\newcommandx\Qrob[2][1=,2=]{\defOperator{#1}{#2}{Q}{}{\lbrack}{\rbrack}{\mathbb}}
\newcommandx\Var[2][1=,2=]{\defOperator{#1}{#2}{V}{ar}{\lbrack}{\rbrack}{\mathbb}}
\newcommandx\Cov[2][1=,2=]{\defOperator{#1}{#2}{C}{ov}{\lbrack}{\rbrack}{\mathbb}}
\newcommandx\Binom[2][1=,2=]{\defOperator{#1}{#2}{B}{}{(}{)}{\mathcal}}
\newcommandx\Gaus[2][1=,2=]{\defOperator{#1}{#2}{N}{}{(}{)}{\mathcal}}
\newcommandx\Wishart[2][1=,2=]{\defOperator{#1}{#2}{W}{ishart}{(}{)}{\mathcal}}
\newcommandx\Likelihood[2][1=,2=]{\defOperator{#1}{#2}{L}{}{(}{)}{\mathcal}}
\newcommandx\logLikelihood[2][1=,2=]{\defOperator{#1}{#2}{\ell}{}{(}{)}{}}
\newcommandx\Information[2][1=,2=]{\defOperator{#1}{#2}{I}{}{(}{)}{\mathcal}}
\newcommandx\Score[2][1=,2=]{\defOperator{#1}{#2}{S}{}{(}{)}{\mathcal}}
\newcommandx\Vois[2][1=,2=]{\defOperator{#1}{#2}{V}{}{(}{)}{\mathcal}}
\newcommandx\IF[2][1=,2=]{\defOperator{#1}{#2}{IF}{}{(}{)}{\mathcal}}
\newcommandx\Ind[1][1=]{\defOperator{}{#1}{1}{}{(}{)}{\mathds}}
\newcommandx\Max[2][1=,2=]{\defUOperator{#1}{#2}{(}{)}{min}}
\newcommandx\Min[2][1=,2=]{\defUOperator{#1}{#2}{(}{)}{max}}
\newcommandx\argMax[2][1=,2=]{\defUOperator{#1}{#2}{(}{)}{argmax}}
\newcommandx\argMin[2][1=,2=]{\defUOperator{#1}{#2}{(}{)}{argmin}}
\newcommandx\cvD[2][1=D,2=n \rightarrow \infty]{\xrightarrow[#2]{#1}}
\newcommandx\Hypothesis[2][1=,2=]{
\ifthenelse{\isempty{#1}}{
\mathcal{H}
}{
\ifthenelse{\isempty{#2}}{
\mathcal{H}_{#1}
}{
\mathcal{H}^{(#2)}_{#1}
}
}
}
\newcommandx\dpartial[4][1=,2=,3=,4=\partial]{
\ifthenelse{\isempty{#3}}{
\frac{#4 #1}{#4 #2}
}{
\left.\frac{#4 #1}{#4 #2}\right\rvert_{#3}
}
}
\newcommandx\dTpartial[3][1=,2=,3=]{\dpartial[#1][#2][#3][d]}
\newcommandx\ddpartial[3][1=,2=,3=]{
\ifthenelse{\isempty{#3}}{
\frac{\partial^{2} #1}{\partial #2^2}
}{
\frac{\partial^2 #1}{\partial #2\partial #3}
}
}
\newcommand\Real{\mathbb{R}}
\newcommand\Rational{\mathbb{Q}}
\newcommand\Natural{\mathbb{N}}
\newcommand\trans[1]{{#1}^\intercal}%\newcommand\trans[1]{{\vphantom{#1}}^\top{#1}}
\newcommand{\independent}{\mathrel{\text{\scalebox{1.5}{$\perp\mkern-10mu\perp$}}}}
\newcommand\half{\frac{1}{2}}
\newcommand\normMax[1]{\left|\left|#1\right|\right|_{max}}
\newcommand\normTwo[1]{\left|\left|#1\right|\right|_{2}}
\newcommand\Veta{\boldsymbol{\eta}}
\newcommand\VX{\mathbf{X}}
\author{Brice Ozenne}
\date{\today}
\title{Performing GPC in a paired design}
\hypersetup{
 colorlinks=true,
 pdfauthor={Brice Ozenne},
 pdftitle={Performing GPC in a paired design},
 pdfkeywords={},
 pdfsubject={},
 pdfcreator={Emacs 29.4 (Org mode 9.6.15)},
 pdflang={English}
 }
\begin{document}

\maketitle
This vignette describes how to use Generalized Pairwise comparisons
(GPC) in a paired design. This for instance corresponds to the
Diabetic Retinopathy Study (DRS) contained in the survival \Rlogo
package where 197 patients had one of their eye randomized to laser
treatment while the other did not receive any treatment:
\begin{lstlisting}[language=r,label= ,caption= ,captionpos=b,numbers=none]
data(diabetic, package = "survival")
head(diabetic)
\end{lstlisting}

\begin{verbatim}
  id laser age   eye trt risk  time status
1  5 argon  28  left   0    9 46.23      0
2  5 argon  28 right   1    9 46.23      0
3 14 xenon  12  left   1    8 42.50      0
4 14 xenon  12 right   0    6 31.30      1
5 16 xenon   9  left   1   11 42.27      0
6 16 xenon   9 right   0   11 42.27      0
\end{verbatim}


The outcome was time to blindness (visual acuity drop below a certain
threshold). In the real study \texttt{status} equal to 0 mixes death and
censoring (due to drop-out or end of study) but this complication will
be neglected here for simplicity.


\bigskip

We will replicate some of the analyzes presented in
\cite{matsouaka2022robust}. In this paper they split the dataset into
juvenile and adult patients:
\begin{lstlisting}[language=r,label= ,caption= ,captionpos=b,numbers=none]
diabetic$juvenile <- diabetic$age <= 19
library(LMMstar)
summarize(age ~ juvenile, data = diabetic[!duplicated(diabetic$id),])
\end{lstlisting}

\begin{verbatim}
  juvenile observed missing     mean        sd min q1 median    q3 max
1    FALSE       83       0 35.30120 11.242054  20 25     34 45.00  58
2     TRUE      114       0 10.21053  4.713892   1  7     10 13.75  19
\end{verbatim}


and we will focus on the juvenile patients:
\begin{lstlisting}[language=r,label= ,caption= ,captionpos=b,numbers=none]
diabeticJ <- diabetic[diabetic$juvenile,]
\end{lstlisting}

\clearpage

\section{Wald methods (Gehan scoring rule)}
\label{sec:org8fd5760}

To mimic the methodology underlying the results presented in Table 1
of \cite{matsouaka2022robust}, we perform GPC stratified by patient
using the Gehan scoring rule:
\begin{lstlisting}[language=r,label= ,caption= ,captionpos=b,numbers=none]
library(BuyseTest)
e.BTjuv <- BuyseTest(trt ~ tte(time,status) + strata(id), 
                     data = diabeticJ, trace = FALSE,
                     scoring.rule =  "Gehan")
model.tables(e.BTjuv, percentage = FALSE)
\end{lstlisting}

\begin{verbatim}
  endpoint total favorable unfavorable neutral uninf     Delta   lower.ci  upper.ci    p.value
1     time   114        39          21       3    51 0.1578947 0.02591623 0.2844633 0.01922741
\end{verbatim}


Indeed this scoring rule does not involve any extra-modeling, only
evaluating the patient specific net benefit and averaging them:
\begin{lstlisting}[language=r,label= ,caption= ,captionpos=b,numbers=none]
mean(coef(e.BTjuv, strata = TRUE))
\end{lstlisting}

\begin{verbatim}
[1] 0.1578947
\end{verbatim}


\cite{matsouaka2022robust} propose to estimate the standard error as:
\begin{lstlisting}[language=r,label= ,caption= ,captionpos=b,numbers=none]
N <- nobs(e.BTjuv)["pairs"]
pw <- coef(e.BTjuv, statistic = "favorable")
pl <- coef(e.BTjuv, statistic = "unfavorable")
sqrt((pw + pl - (pw - pl)^2)/N)
\end{lstlisting}

\begin{verbatim}
     pairs 
0.06631828
\end{verbatim}


which matches what \texttt{BuyseTest} output:
\begin{lstlisting}[language=r,label= ,caption= ,captionpos=b,numbers=none]
confint(e.BTjuv)
\end{lstlisting}

\begin{verbatim}
      estimate         se   lower.ci  upper.ci null    p.value
time 0.1578947 0.06631828 0.02791329 0.2878762    0 0.01727214
\end{verbatim}


By default \texttt{confint} uses a hyperbolic tangent unless the argument
\texttt{transform} is set to \texttt{FALSE}.

\bigskip

\uline{Note:} naively one may think to estimate the standard error as:
\begin{lstlisting}[language=r,label= ,caption= ,captionpos=b,numbers=none]
sqrt(var(coef(e.BTjuv, strata = TRUE))/N)
\end{lstlisting}

\begin{verbatim}
     pairs 
0.06661108
\end{verbatim}


This is equivalent (in large samples to the previous formula). Indeed:
\begin{align*}
&\Prob[X>Y] + \Prob[Y>X] - (\Prob[X>Y] - \Prob[Y>X])^2 \\
=& \Prob[X>Y] + \Prob[Y>X] - \Prob[X>Y]^ - \Prob[Y>X]^2 + 2 \Prob[X>Y] \Prob[Y>X] \\
=& \Prob[X>Y](1-\Prob[X>Y]) + \Prob[Y>X](1-\Prob[Y>X]) + 2 \Prob[X>Y] \Prob[Y>X] \\
=& \Prob[X>Y](1-\Prob[X>Y]) + \Prob[Y>X](1-\Prob[Y>X]) \\
 & - 2 (0 - \Prob[X>Y] \Prob[Y>X] - \Prob[X>Y] \Prob[Y>X] + \Prob[X>Y] \Prob[Y>X] \\
=& \Var\left[\Ind[X>Y]\right] + \Var\left[\Ind[X<Y]\right] - 2 \Cov\left(\Ind[X>Y],\Ind[X<Y]\right) \\
=& \Var\left[\Ind[X>Y]-\Ind[X<Y]\right] \\
\end{align*}

There is only a factor \texttt{N/(N-1)} difference between the two:
\begin{lstlisting}[language=r,label= ,caption= ,captionpos=b,numbers=none]
sqrt(var(coef(e.BTjuv, strata = TRUE))/N) * sqrt((N-1)/N)
\end{lstlisting}

\begin{verbatim}
     pairs 
0.06631828
\end{verbatim}


\clearpage

\section{MOVER method (Gehan scoring rule)}
\label{sec:org1938659}

In fact the results of table 1 were obtained using MOVER
approach. Thye can be retrieved with a dedicated function processing
the output of the \texttt{BuyseTest} function:

\begin{lstlisting}[language=r,label= ,caption= ,captionpos=b,numbers=none,basicstyle=\ttfamily\tiny]
runMover <- function(object, level = 0.95, p.value = TRUE, type = "Wilson", tol = 1e-6){
    
    ## ** normalize user input
    q.low <- qnorm((1-level)/2)
    q.high <- qnorm(1-(1-level)/2)
    type <- match.arg(type, c("Wilson","Agresti-Coull"))
    
    ## ** normalize user input
    N <- nobs(object)["pairs"]
    pi_w <- coef(object, statistic = "favorable")
    pi_l <- coef(object, statistic = "unfavorable")

    if(level == 1){
        DeltaL <- -1
        DeltaU <- 1
    }else{
        Ntilde <- N + q.high
        pitilde_w <- (N*pi_w+0.5*q.high^2)/Ntilde
        pitilde_l <- (N*pi_l+0.5*q.high^2)/Ntilde

        if(any(c(pi_w,pi_l)<tol) || any(c(pi_w,pi_l)>1-tol)){
            cor_wl <- 0
        }else{
            cor_wl <- - pi_w * pi_l / sqrt(pi_w * (1 - pi_w) * pi_l * (1 - pi_l))
        }

        ## Wilson
        if(type == "Wilson"){
            L_w <- pitilde_w + 0.5/Ntilde * q.low * sqrt(q.low^2 + 4*N*pi_w*(1-pi_w))
            U_w <- pitilde_w + 0.5/Ntilde * q.high * sqrt(q.high^2 + 4*N*pi_w*(1-pi_w))
            L_l <- pitilde_l + 0.5/Ntilde * q.low * sqrt(q.low^2 + 4*N*pi_l*(1-pi_l))
            U_l <- pitilde_l + 0.5/Ntilde * q.high * sqrt(q.high^2 + 4*N*pi_l*(1-pi_l))
        }else if(type == "Agresti-Coull"){
            L_w <- pitilde_w + q.low * sqrt(pitilde_w*(1-pitilde_w)/Ntilde)
            U_w <- pitilde_w + q.high * sqrt(pitilde_w*(1-pitilde_w)/Ntilde)
            L_l <- pitilde_l + q.low * sqrt(pitilde_l*(1-pitilde_l)/Ntilde)
            U_l <- pitilde_l + q.high * sqrt(pitilde_l*(1-pitilde_l)/Ntilde)
        }

        DeltaL <- (pi_w-pi_l) - sqrt((pi_w - L_w)^2 + (U_l - pi_l)^2 - 2 * cor_wl * (pi_w - L_w) * (U_l - pi_l))
        DeltaU <- (pi_w-pi_l) + sqrt((U_w - pi_w)^2 + (pi_l - L_l)^2 - 2 * cor_wl * (U_w - pi_w) * (pi_l - L_l))
    }

    ## p-value
    if(p.value){
        if(abs(DeltaL)<tol | abs(DeltaU)<tol){ ## CI touch 0
            p <- 1-level
        }else if(DeltaL>0 && DeltaU>0){ ## CI do not overlap 0: p>level
            res.search <- uniroot(f = function(mylevel){runMover(object, level = mylevel, p.value = FALSE)["lower"]},
                                  lower = level, upper = 1)
        }else if(DeltaL<0 && DeltaU<0){ ## CI do not overlap 0: p>level
            res.search <- uniroot(f = function(mylevel){runMover(object, level = mylevel, p.value = FALSE)["upper"]},
                                  lower = level, upper = 1)
        }else if(DeltaL<0 && DeltaU>0){ ## CI overlaps 0: p<level
            res.search <- uniroot(f = function(mylevel){
                tempo <- runMover(object, level = mylevel, p.value = FALSE)[c("lower","upper")]
                return(tempo[which.min(abs(tempo))])
            }, lower = 0, upper = level)
        }
        p.value <- 1-res.search$root            
    }else{
        p.value <- NA
    }

    ## export
    out <- c(estimate = pi_w-pi_l, lower = unname(DeltaL), upper = unname(DeltaU), pvalue = p.value)
    return(out)
}
\end{lstlisting}


\begin{lstlisting}[language=r,label= ,caption= ,captionpos=b,numbers=none]
runMover(e.BTjuv, level = 0.95)
\end{lstlisting}
\begin{verbatim}
  estimate      lower      upper     pvalue 
0.15789474 0.02525513 0.28805176 0.02048651
\end{verbatim}



\clearpage

\section{Wald methods (Peron scoring rule)}
\label{sec:org20ae582}

To better account for censoring one could use the Peron scoring rule
where the survival is estimated across all subjects within a treatment
group. One has to specify the survival model, otherwise, the BuyseTest
function will estimate a treatmnet and strata specific survival curve
which is impossible to perform here. The \texttt{model.tte} argument can be
used to specify such survival model:
\begin{lstlisting}[language=r,label= ,caption= ,captionpos=b,numbers=none]
library(prodlim)
e.BTjuv2 <- BuyseTest(trt ~ tte(time,status) + strata(id), 
                      data = diabeticJ, trace = FALSE,
                      model.tte = prodlim(Hist(time,status)~ trt, data = diabeticJ))
model.tables(e.BTjuv2, percentage = FALSE)
\end{lstlisting}

\begin{verbatim}
  endpoint total favorable unfavorable neutral    uninf    Delta   lower.ci  upper.ci     p.value
1     time   114  47.36525    24.29552       3 39.33923 0.202366 0.05045454 0.3451254 0.009329589
\end{verbatim}


Ignoring the uncertainty of the survival model, the standard would be:
\begin{lstlisting}[language=r,label= ,caption= ,captionpos=b,numbers=none]
c(sqrt(var(coef(e.BTjuv2, strata = TRUE))/N),
  sqrt(var(coef(e.BTjuv2, strata = TRUE))/N) * sqrt((N-1)/N)
  )
\end{lstlisting}

\begin{verbatim}
     pairs      pairs 
0.06595510 0.06566518
\end{verbatim}


depending on whether a small sample correction is used or not. This
matches the output of \texttt{BuyseTest} when ignoring the uncertainty of the
survival model:
\begin{lstlisting}[language=r,label= ,caption= ,captionpos=b,numbers=none]
model.tte <- prodlim(Hist(time,status)~ trt, data = diabeticJ)
attr(model.tte, "iidNuisance") <- FALSE
confint(BuyseTest(trt ~ tte(time,status) + strata(id), 
                  data = diabeticJ, trace = FALSE,
                  model.tte = model.tte))
\end{lstlisting}

\begin{verbatim}
     estimate         se   lower.ci  upper.ci null     p.value
time 0.202366 0.06566518 0.07088227 0.3269375    0 0.002726979
\end{verbatim}


\Warning Because the pairwise win and loss score are no more binary, the
previous formula no more simplifies into the formula presented in
\cite{matsouaka2022robust}:
\begin{lstlisting}[language=r,label= ,caption= ,captionpos=b,numbers=none]
pw.peron <- coef(e.BTjuv2, statistic = "favorable")
pl.peron <- coef(e.BTjuv2, statistic = "unfavorable")
sqrt((pw.peron + pl.peron - (pw.peron - pl.peron)^2)/N)
\end{lstlisting}

\begin{verbatim}
     pairs 
0.07179718
\end{verbatim}


\clearpage 

To account for the uncertainty of the survival model, \texttt{BuyseTest}
outputs a slightly higher standard error:
\begin{lstlisting}[language=r,label= ,caption= ,captionpos=b,numbers=none]
confint(e.BTjuv2)
\end{lstlisting}

\begin{verbatim}
     estimate         se   lower.ci  upper.ci null     p.value
time 0.202366 0.07569815 0.05045454 0.3451254    0 0.009329589
\end{verbatim}


This is achieved by considering two sources of uncertainty:
\begin{itemize}
\item average of a finite number of pairs:
\end{itemize}
\begin{lstlisting}[language=r,label= ,caption= ,captionpos=b,numbers=none]
pw.peronS <- coef(e.BTjuv2, statistic = "favorable", strata = TRUE)
pl.peronS <- coef(e.BTjuv2, statistic = "unfavorable", strata = TRUE)
Hterm1 <- (pw.peronS - pl.peronS) - (pw.peron - pl.peron)
\end{lstlisting}

\begin{itemize}
\item propage the uncertainty of the survival model to the net
benefit. Because each pair appear twice (control and treatment) the
impact of removing a pair on the net benefit is stored in the
control and the treated is set to 0:
\end{itemize}
\begin{lstlisting}[language=r,label= ,caption= ,captionpos=b,numbers=none]
Hterm2.obs <- e.BTjuv2@iidNuisance$favorable - e.BTjuv2@iidNuisance$unfavorable
Hterm2.pair <- Hterm2.obs[diabeticJ$trt==0]
table(Hterm2.obs[diabeticJ$trt==1])
\end{lstlisting}

\begin{verbatim}

  0 
114
\end{verbatim}


After rescaling the terms by a factor N (number of pairs, to account
for the pooling) we retrieve the uncertainty output by \texttt{BuyseTest}:
\begin{lstlisting}[language=r,label= ,caption= ,captionpos=b,numbers=none]
c(average = sqrt(crossprod((Hterm1/N))),
  nuisance = sqrt(crossprod((Hterm2.pair/N))),
  all = sqrt(crossprod((Hterm1/N + Hterm2.pair/N))))
\end{lstlisting}

\begin{verbatim}
   average   nuisance        all 
0.06566518 0.02084622 0.07569815
\end{verbatim}



\section*{References}
\label{sec:org0211660}
\begingroup
\renewcommand{\section}[2]{}

\bibliographystyle{apalike}
\bibliography{bibliography}

\endgroup
\end{document}
