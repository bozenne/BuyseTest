% Created 2018-07-09 ma 16:29
% Intended LaTeX compiler: pdflatex
\documentclass{article}

%%%% settings when exporting code %%%% 

\usepackage{listings}
\lstset{
backgroundcolor=\color{white},
basewidth={0.5em,0.4em},
basicstyle=\ttfamily\small,
breakatwhitespace=false,
breaklines=true,
columns=fullflexible,
commentstyle=\color[rgb]{0.5,0,0.5},
frame=single,
keepspaces=true,
keywordstyle=\color{black},
literate={~}{$\sim$}{1},
numbers=left,
numbersep=10pt,
numberstyle=\ttfamily\tiny\color{gray},
showspaces=false,
showstringspaces=false,
stepnumber=1,
stringstyle=\color[rgb]{0,.5,0},
tabsize=4,
xleftmargin=.23in,
emph={anova,apply,class,coef,colnames,colNames,colSums,dim,dcast,for,ggplot,head,if,ifelse,is.na,lapply,list.files,library,logLik,melt,plot,require,rowSums,sapply,setcolorder,setkey,str,summary,tapply},
emphstyle=\color{blue}
}

%%%% packages %%%%%

\usepackage[utf8]{inputenc}
\usepackage[T1]{fontenc}
\usepackage{lmodern}
\usepackage{textcomp}
\usepackage{color}
\usepackage{enumerate}
\usepackage{graphicx}
\usepackage{grffile}
\usepackage{wrapfig}
\usepackage{capt-of}
\usepackage{caption}
\usepackage{rotating}
\usepackage{longtable}
\usepackage{multirow}
\usepackage{multicol}
\usepackage{changes}
\usepackage{pdflscape}
\usepackage{geometry}
\usepackage[normalem]{ulem}
\usepackage{amssymb}
\usepackage{amsmath}
\usepackage{amsfonts}
\usepackage{dsfont}
\usepackage{textcomp}
\usepackage{array}
\usepackage{ifthen}
\usepackage{hyperref}
\usepackage{natbib}
\RequirePackage{fancyvrb}
\DefineVerbatimEnvironment{verbatim}{Verbatim}{fontsize=\small,formatcom = {\color[rgb]{0.5,0,0}}}
\RequirePackage{colortbl} % arrayrulecolor to mix colors
\RequirePackage{setspace} % to modify the space between lines - incompatible with footnote in beamer
\usepackage{authblk} % enable several affiliations (clash with beamer)
\RequirePackage{epstopdf} % to be able to convert .eps to .pdf image files
%
%%%% additional latex commands %%%%
%
\date{}
\title{Comparaison de méthodes d'inférence dans le cadre des comparaisons par paires généralisées}
\hypersetup{
 colorlinks=true,
 citecolor=[rgb]{0,0.5,0},
 urlcolor=[rgb]{0,0,0.5},
 linkcolor=[rgb]{0,0,0.5},
 pdfauthor={Brice Ozenne},
 pdftitle={Comparaison de méthodes d'inférence dans le cadre des comparaisons par paires généralisées},
 pdfkeywords={},
 pdfsubject={},
 pdfcreator={Emacs 25.2.1 (Org mode 9.0.4)},
 pdflang={English}
 }
\begin{document}

\author{}
\maketitle{}

Les comparaisons par paires généralisées (\citep{buyse2010generalized},
\citep{peron2018extension}) est une méthode d'analyse statistique
permettant de comparer un groupe de patients ayant reçu un traitement
à un groupe contrôle en considérant plusieurs critères de
jugement. Cette méthode permet d'estimer le bénéfice net en faveur du
traitement valant 1 lorsque le traitement a été favorable pour tous
les patients, -1 lorsqu'il a toujours été défavorable, et 0 lorsque
l'effet traitement au niveau groupe est nul. 

\bigskip

Afin de pouvoir tester si l'effet traitement est favorable or de
calculer son intervalle de confiance, il est nécessaire de connaître la
distribution statistique du bénéfice net en faveur du traitement. Pour
cela plusieurs méthodes sont envisageables:
\begin{itemize}
\item méthode de permutation \citep{buyse2010generalized}
\item techniques de bootstrap \citep{dong2016generalized}
\item utilisation de la distribution asymptotique \citep{luo2015alternative}, \citep{bebu2015large}
\end{itemize}

L'objectif du stage est de comparer ces méthodes en terme d'hypothèses
nécessaires à leur validité, contrôle de l'erreur de type 1 en
échantillons finis, et ressources informatiques nécessaires à leur
utilisation. L'évaluation de ces méthodes se fera en relation avec la
librairie \texttt{BuyseTest} du logiciel R qui implémente les comparaisons par
paires généralisées. Cette librairie permet d'utiliser une méthode de
permutation ainsi que deux types de boostrap. L'estimation de la
distribution asymptotique est actuellement en cours de développement;
l'étudiant pourra prendre part à ce développement s'il souhaite un
sujet de stage plus théorique.

\bigskip

\textbf{Durée}: 6 mois

\textbf{Début du stage}: ??

\textbf{Laboratoire de rattachement}: Service de biostatistiques des hospices de Lyon.

\textbf{Profil}: Le candidat devra être en formation au sein d’un Master 2 à
dominante statistique ou dans école d’ingénieur avec une spécialisation en statistique.

\textbf{Encadrants}: Julien Péron (\ldots{})  Brice Ozenne (Post doctorant au
département de biostatistique de l'université de Copenhague, en charge
du dévelopement et de la maintenance du code du package \texttt{BuyseTest})

\textbf{Contact}: Julien Péron (julien.peron@chu-lyon.fr) 

\bigskip

\bibliographystyle{apalike}
\bibliography{bibliography}
\end{document}