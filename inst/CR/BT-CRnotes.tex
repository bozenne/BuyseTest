% Created 2018-09-08 lør 20:02
% Intended LaTeX compiler: pdflatex
\documentclass{article}

%%%% settings when exporting code %%%% 

\usepackage{listings}
\lstset{
backgroundcolor=\color{white},
basewidth={0.5em,0.4em},
basicstyle=\ttfamily\small,
breakatwhitespace=false,
breaklines=true,
columns=fullflexible,
commentstyle=\color[rgb]{0.5,0,0.5},
frame=single,
keepspaces=true,
keywordstyle=\color{black},
literate={~}{$\sim$}{1},
numbers=left,
numbersep=10pt,
numberstyle=\ttfamily\tiny\color{gray},
showspaces=false,
showstringspaces=false,
stepnumber=1,
stringstyle=\color[rgb]{0,.5,0},
tabsize=4,
xleftmargin=.23in,
emph={anova,apply,class,coef,colnames,colNames,colSums,dim,dcast,for,ggplot,head,if,ifelse,is.na,lapply,list.files,library,logLik,melt,plot,require,rowSums,sapply,setcolorder,setkey,str,summary,tapply},
emphstyle=\color{blue}
}

%%%% packages %%%%%

\usepackage[utf8]{inputenc}
\usepackage[T1]{fontenc}
\usepackage{lmodern}
\usepackage{textcomp}
\usepackage{color}
\usepackage{enumerate}
\usepackage{graphicx}
\usepackage{grffile}
\usepackage{wrapfig}
\usepackage{rotating}
\usepackage{longtable}
\usepackage{multirow}
\usepackage{multicol}
\usepackage{changes}
\usepackage{pdflscape}
\usepackage{geometry}
\usepackage[normalem]{ulem}
\usepackage{amssymb}
\usepackage{amsmath}
\usepackage{amsfonts}
\usepackage{dsfont}
\usepackage{textcomp}
\usepackage{array}
\usepackage{ifthen}
\usepackage{hyperref}
\usepackage{natbib}
\RequirePackage{fancyvrb}
\DefineVerbatimEnvironment{verbatim}{Verbatim}{fontsize=\small,formatcom = {\color[rgb]{0.5,0,0}}}
\RequirePackage{colortbl} % arrayrulecolor to mix colors
\RequirePackage{setspace} % to modify the space between lines - incompatible with footnote in beamer
\usepackage{authblk} % enable several affiliations (clash with beamer)
\RequirePackage{epstopdf} % to be able to convert .eps to .pdf image files
%
%%%% additional latex commands %%%%
%
\RequirePackage{amsmath}
\RequirePackage{algorithm}
\RequirePackage[noend]{algpseudocode}
\RequirePackage{ifthen}
\RequirePackage{xspace} % space for newcommand macro
\RequirePackage{xifthen}
\RequirePackage{xargs}
\RequirePackage{dsfont}
\RequirePackage{amsmath,stmaryrd,graphicx}
\RequirePackage{prodint} % product integral symbol (\PRODI)
\RequirePackage{amsthm}
\newtheorem{theorem}{Theorem}
\newtheorem{lemma}[theorem]{Lemma}
\newcommand\defOperator[7]{%
\ifthenelse{\isempty{#2}}{
\ifthenelse{\isempty{#1}}{#7{#3}#4}{#7{#3}#4 \left#5 #1 \right#6}
}{
\ifthenelse{\isempty{#1}}{#7{#3}#4_{#2}}{#7{#3}#4_{#1}\left#5 #2 \right#6}
}
}
\newcommand\defUOperator[5]{%
\ifthenelse{\isempty{#1}}{
#5\left#3 #2 \right#4
}{
\ifthenelse{\isempty{#2}}{\underset{#1}{\operatornamewithlimits{#5}}}{
\underset{#1}{\operatornamewithlimits{#5}}\left#3 #2 \right#4}
}
}
\newcommand{\defBoldVar}[2]{
\ifthenelse{\equal{#2}{T}}{\boldsymbol{#1}}{\mathbf{#1}}
}
\newcommandx\Cov[2][1=,2=]{\defOperator{#1}{#2}{C}{ov}{\lbrack}{\rbrack}{\mathbb}}
\newcommandx\Esp[2][1=,2=]{\defOperator{#1}{#2}{E}{}{\lbrack}{\rbrack}{\mathbb}}
\newcommandx\Prob[2][1=,2=]{\defOperator{#1}{#2}{P}{}{\lbrack}{\rbrack}{\mathbb}}
\newcommandx\Qrob[2][1=,2=]{\defOperator{#1}{#2}{Q}{}{\lbrack}{\rbrack}{\mathbb}}
\newcommandx\Var[2][1=,2=]{\defOperator{#1}{#2}{V}{ar}{\lbrack}{\rbrack}{\mathbb}}
\newcommandx\Binom[2][1=,2=]{\defOperator{#1}{#2}{B}{}{(}{)}{\mathcal}}
\newcommandx\Gaus[2][1=,2=]{\defOperator{#1}{#2}{N}{}{(}{)}{\mathcal}}
\newcommandx\Wishart[2][1=,2=]{\defOperator{#1}{#2}{W}{ishart}{(}{)}{\mathcal}}
\newcommandx\Likelihood[2][1=,2=]{\defOperator{#1}{#2}{L}{}{(}{)}{\mathcal}}
\newcommandx\Information[2][1=,2=]{\defOperator{#1}{#2}{I}{}{(}{)}{\mathcal}}
\newcommandx\Score[2][1=,2=]{\defOperator{#1}{#2}{S}{}{(}{)}{\mathcal}}
\newcommandx\Vois[2][1=,2=]{\defOperator{#1}{#2}{V}{}{(}{)}{\mathcal}}
\newcommandx\IF[2][1=,2=]{\defOperator{#1}{#2}{IF}{}{(}{)}{\mathcal}}
\newcommandx\Ind[1][1=]{\defOperator{}{#1}{1}{}{(}{)}{\mathds}}
\newcommandx\Max[2][1=,2=]{\defUOperator{#1}{#2}{(}{)}{min}}
\newcommandx\Min[2][1=,2=]{\defUOperator{#1}{#2}{(}{)}{max}}
\newcommandx\argMax[2][1=,2=]{\defUOperator{#1}{#2}{(}{)}{argmax}}
\newcommandx\argMin[2][1=,2=]{\defUOperator{#1}{#2}{(}{)}{argmin}}
\newcommandx\cvD[2][1=D,2=n \rightarrow \infty]{\xrightarrow[#2]{#1}}
\newcommandx\Hypothesis[2][1=,2=]{
\ifthenelse{\isempty{#1}}{
\mathcal{H}
}{
\ifthenelse{\isempty{#2}}{
\mathcal{H}_{#1}
}{
\mathcal{H}^{(#2)}_{#1}
}
}
}
\newcommandx\dpartial[4][1=,2=,3=,4=\partial]{
\ifthenelse{\isempty{#3}}{
\frac{#4 #1}{#4 #2}
}{
\left.\frac{#4 #1}{#4 #2}\right\rvert_{#3}
}
}
\newcommandx\dTpartial[3][1=,2=,3=]{\dpartial[#1][#2][#3][d]}
\newcommandx\ddpartial[3][1=,2=,3=]{
\ifthenelse{\isempty{#3}}{
\frac{\partial^{2} #1}{\left( \partial #2\right)^2}
}{
\frac{\partial^2 #1}{\partial #2\partial #3}
}
}
\newcommand\Real{\mathbb{R}}
\newcommand\Rational{\mathbb{Q}}
\newcommand\Natural{\mathbb{N}}
\newcommand\trans[1]{{#1}^\intercal}%\newcommand\trans[1]{{\vphantom{#1}}^\top{#1}}
\newcommand{\independent}{\mathrel{\text{\scalebox{1.5}{$\perp\mkern-10mu\perp$}}}}
\newcommand\half{\frac{1}{2}}
\newcommand\normMax[1]{\left|\left|#1\right|\right|_{max}}
\newcommand\normTwo[1]{\left|\left|#1\right|\right|_{2}}
\author{Brice Ozenne, Julien Péron}
\date{\today}
\title{Explicit formula for the net benefit}
\hypersetup{
 colorlinks=true,
 citecolor=[rgb]{0,0.5,0},
 urlcolor=[rgb]{0,0,0.5},
 linkcolor=[rgb]{0,0,0.5},
 pdfauthor={Brice Ozenne, Julien Péron},
 pdftitle={Explicit formula for the net benefit},
 pdfkeywords={},
 pdfsubject={},
 pdfcreator={Emacs 27.0.50 (Org mode 9.0.4)},
 pdflang={English}
 }
\begin{document}

\maketitle
\tableofcontents

\clearpage

\section{Parameter of interest}
\label{sec:org7d731bb}

Let consider two independent real valued random variables \(X\) and \(Y\).
We are interested in:
\begin{align*}
\Delta = \Prob[Y>X] - \Prob[X>Y]
\end{align*}

\bigskip

In the examples we will use a sample size of:
\lstset{language=r,label= ,caption= ,captionpos=b,numbers=none}
\begin{lstlisting}
n <- 1e4
\end{lstlisting}

and use the following R packages
\lstset{language=r,label= ,caption= ,captionpos=b,numbers=none}
\begin{lstlisting}
library(BuyseTest)
library(riskRegression)
library(survival)
\end{lstlisting}

\clearpage

\section{Binary variable}
\label{sec:org8d07799}
\subsection{Relationship between \(\Delta\) and the prevalence}
\label{sec:org908c24e}
\begin{align*}
\Prob[Y>X] = \Prob[Y=1,X=0]
\end{align*}
Using the independence between \(Y\) and \(X\):
\begin{align*}
\Prob[Y>X] = \Prob[Y=1]\Prob[X=0] = \Prob[Y=1](1-\Prob[X=1]) = \Prob[Y=1] - \Prob[Y=1]\Prob[X=1]
\end{align*}
By symmetry:
\begin{align*}
\Prob[X>Y] = \Prob[X=1] - \Prob[Y=1]\Prob[X=1]
\end{align*}
So 
\begin{align*}
\Delta = \Prob[Y=1] - \Prob[X=0]
\end{align*}

\subsection{In R}
\label{sec:orgf4c6402}
Settings:
\lstset{language=r,label= ,caption= ,captionpos=b,numbers=none}
\begin{lstlisting}
prob1 <- 0.4
prob2 <- 0.2
\end{lstlisting}

Simulate data:
\lstset{language=r,label= ,caption= ,captionpos=b,numbers=none}
\begin{lstlisting}
set.seed(10)
df <- rbind(data.frame(tox = rbinom(n, prob = prob1, size = 1), group = "C"),
	    data.frame(tox = rbinom(n, prob = prob2, size = 1), group = "T"))
\end{lstlisting}

Buyse test:
\lstset{language=r,label= ,caption= ,captionpos=b,numbers=none}
\begin{lstlisting}
BuyseTest(group ~ bin(tox), data = df, method.inference = "none", trace = 0)
\end{lstlisting}
\begin{verbatim}
endpoint threshold   delta   Delta
     tox       0.5 -0.1981 -0.1981
\end{verbatim}

Expected:
\lstset{language=r,label= ,caption= ,captionpos=b,numbers=none}
\begin{lstlisting}
prob2 - prob1
\end{lstlisting}

\begin{verbatim}
[1] -0.2
\end{verbatim}

\clearpage

\section{Continuous variable}
\label{sec:org7d7bd4f}
\subsection{Relationship between \(\Delta\) and Cohen's d}
\label{sec:org5abbb60}
Let's consider two independent normally distributed variables with common variance:
\begin{itemize}
\item \(X \sim \Gaus[\mu_X,\sigma^2]\)
\item \(Y \sim \Gaus[\mu_Y,\sigma^2]\)
\end{itemize}
Denoting \(d = \frac{\mu_Y-\mu_X}{\sigma}\): 
\begin{itemize}
\item \(X^* \sim \Gaus[0,1]\)
\item \(Y^* \sim \Gaus[d,1]\)
\end{itemize}
\begin{align*}
\Prob[Y>X] &= \Esp \left[ \Ind[Y>X] \right] = \Esp\left[ \Ind[Y*>X*] \right] = \Esp\left[ \Ind[Z>0] \right]
\end{align*}
where \(Z \sim \Gaus[d,2]\) so \(\Prob[Y>X] = \Phi(\frac{d}{\sqrt{2}})\)

By symmetry
\begin{align*}
\Delta = 2*\Phi(\frac{d}{\sqrt{2}})-1
\end{align*}

\subsection{In R}
\label{sec:orgd0c6a49}

Settings:
\lstset{language=r,label= ,caption= ,captionpos=b,numbers=none}
\begin{lstlisting}
meanX <- 0
meanY <- 2
sdXY <- 1
\end{lstlisting}

Simulate data:
\lstset{language=r,label= ,caption= ,captionpos=b,numbers=none}
\begin{lstlisting}
set.seed(10)
df <- rbind(data.frame(tox = rnorm(n, mean = meanX, sd = sdXY), group = "C"),
	    data.frame(tox = rnorm(n, mean = meanY, sd = sdXY), group = "T"))
\end{lstlisting}

Buyse test:
\lstset{language=r,label= ,caption= ,captionpos=b,numbers=none}
\begin{lstlisting}
BuyseTest(group ~ cont(tox), data = df, method.inference = "none", trace = 0)
\end{lstlisting}

\begin{verbatim}
endpoint threshold  delta  Delta
     tox     1e-12 0.8359 0.8359
\end{verbatim}

Expected:
\lstset{language=r,label= ,caption= ,captionpos=b,numbers=none}
\begin{lstlisting}
d <- (meanY-meanX)/sdXY
2*pnorm(d/sqrt(2))-1
\end{lstlisting}

\begin{verbatim}
[1] 0.8427008
\end{verbatim}

\clearpage

\section{Survival}
\label{sec:orgb02f16d}
\subsection{Relationship between \(\Delta\) and the hazard ratio}
\label{sec:org121a536}
For a given cumulative density function \(F(x)\) and a corresponding
probability density function \(f(x)\) we define the hazard by:
\begin{align*}
\lambda(t) &=  \left. \frac{\Prob[t\leq T \leq t+h|T\geq t]}{h}\right|_{h \rightarrow 0^+} \\
&= \left. \frac{\Prob[t\leq T \leq t+h]}{\Prob[T\geq t]h}\right|_{h \rightarrow 0^+} \\
&= \frac{f(t)}{1-F(t)}
\end{align*}

\bigskip

Let now consider two times to events following an exponential distribution:
\begin{itemize}
\item \(X \sim Exp(\alpha_1)\). The corresponding hazard function is \(\lambda(t)=\alpha_1\).
\item \(Y \sim Exp(\alpha_2)\). The corresponding hazard function is \(\lambda(t)=\alpha_2\).
\end{itemize}
So the hazad ratio is \(HR = \frac{\lambda_2}{\lambda_1}\). Note that if we use a cox model we will have:
\begin{align*}
\lambda(t) = \lambda_0(t) \exp(\beta \Ind[group])
\end{align*}
where \(\exp(\beta)\) is the hazard ratio.

\bigskip

\begin{align*}
\Prob[Y>X] &= \int_{0}^{\infty}\alpha_1 \exp(-\alpha_1 x)  \int_x^{\infty} \alpha_2 \exp(-\alpha_2 y) dy dx \\
&= \int_{0}^{\infty}\alpha_1 \exp(-\alpha_1 x)  [ \exp(-\alpha_2 y) ]_{\infty}^{x} dx \\
&= \int_{0}^{\infty}\alpha_1 \exp(-\alpha_1 x) \exp(-\alpha_2 x) dx \\
&= \frac{\alpha_1}{\alpha_1+\alpha_2} [\exp(-(\alpha_1+\alpha_2) x)]_{\infty}^{0} \\
&= \frac{\alpha_1}{\alpha_1+\alpha_2}\\
&= \frac{1}{1+HR}\\
\end{align*}

So:
\begin{align*}
\Delta = 2\frac{1}{1+HR}-1 = \frac{1-HR}{1+HR}
\end{align*}

\subsection{Scoring rule in presence of censoring}
\label{sec:orgd81bf6d}
Let's consider the following random variables:  
\begin{itemize}
\item \(X\) the time to the occurrence of the event of interest in the treatment group.
\item \(C_X\) the censoring time in the treatment group.
\item \(X^* = X \wedge C_X\) the observed event time in the treatment group.
\item \(\varepsilon_X = \Ind[X \leq C_X]\) the event time indicator in the treatment group.
\item \(Y\) the time to the occurrence of the event of interest in the control group.
\item \(C_Y\) the censoring time in the control group.
\item \(Y^* = Y \wedge C_Y\) the observed event time in the control group.
\item \(\varepsilon_Y = \Ind[Y \leq C_Y]\) the event time indicator in the control group.
\end{itemize}

We observe one realization \(\left(x^*, y^*, e_X, e_Y \right)\) of the
random variables \(\left(X^*, Y^*, \varepsilon_X, \varepsilon_Y
\right)\). We use the short notation \(x \wedge y = min(x,y)\) and \(x \vee y = max(x,y)\).

\subsubsection{Case: \(e_X=0,e_Y=1\)}
\label{sec:orgadc882c}

\textbf{Probability in favor of the treatment}:
\begin{align*}
\Prob[x \geq y + \tau | x \geq x^*, y = y^*] 
&= \frac{ \Prob[x \geq y^* + \tau, x \geq x^*] }{ \Prob[x \geq x^*]} \\
&= \frac{ \Prob[x \geq max(y^* + \tau,x^*)] }{\Prob[x \geq x^*]} \\
&= \frac{ S_X(y^* + \tau \vee x^*)}{S_X(x^*)}
\end{align*}

In the case where \(x^* < y^* + \tau\), we need an estimate of
\(S_X(y^* + \tau)\) to compute the probability in favor of the
treatment. If we can only have an estimate of \(S_X\) up to
\(x_{max} < y^* + \tau\) then we can use the following inequality:
\begin{align*}
S_X(y^* + \tau) &\geq 0 \\
\Prob[x \geq y + \tau | x \geq x^*, y = y^*] &\geq 0 \\
\end{align*}

\textbf{Probability in favor of the control}:

\begin{align*}
\Prob[y \geq x + \tau | x \geq x^*, y = y^*] 
&= 1 - \frac{ \Prob[x \geq y^* - \tau, x \geq x^*] }{ \Prob[x \geq x^*]} \\
&= 1 - \frac{ \Prob[x \geq max(y^* - \tau,x^*)] }{\Prob[x \geq x^*]} \\
&= 1 - \frac{ S_X(y^* - \tau \vee x^*)}{S_X(x^*)}
\end{align*}

In the case where \(x^* < y^* - \tau\), we need an estimate of
\(S_X(y^* - \tau)\) to compute the probability in favor of the
control. If we can only have an estimate of \(S_X\) up to
\(x_{max} < y^* - \tau\) then we can use the following inequality:
\begin{align*}
S_X(x_{max}) &\geq S_X(y^* - \tau) \\
\Prob[x \geq y - \tau | x \geq x^*, y = y^*] &\geq 1 - \frac{ S_X(x_{max})}{S_X(x^*)} \\
\end{align*}

\textbf{Probability of being neutral}:

\begin{align*}
\Prob[|x-y| \leq \tau | x \geq x^*, y = y^*] 
&= 1-\Prob[x \geq y + \tau | x \geq x^*, y = y^*]-\Prob[y \geq x + \tau | x \geq x^*, y = y^*]  \\
&= \frac{ S_X(y^* - \tau \vee x^*) - S_X(y^* + \tau \vee x^*)}{S_X(x^*)}
\end{align*}

If \(x_{max} > y^* - \tau\) then 
\begin{align*}
\Prob[|x-y| \leq \tau | x \geq x^*, y = y^*] \geq \frac{ S_X(y^* - \tau) - S_X(x_{max})}{S_X(x^*)}
\end{align*}

\subsubsection{Data driven estimation of the expected survival in each treatment group}
\label{sec:org3309535}
\subsection{In R}
\label{sec:orga125895}

Settings:
\lstset{language=r,label= ,caption= ,captionpos=b,numbers=none}
\begin{lstlisting}
alphaX <- 2
alphaY <- 1
\end{lstlisting}

Simulate data:
\lstset{language=r,label= ,caption= ,captionpos=b,numbers=none}
\begin{lstlisting}
set.seed(10)
df <- rbind(data.frame(time = rexp(n, rate = alphaX), group = "C", event = 1),
	    data.frame(time = rexp(n, rate = alphaY), group = "T", event = 1))
\end{lstlisting}

Buyse test:
\lstset{language=r,label= ,caption= ,captionpos=b,numbers=none}
\begin{lstlisting}
BuyseTest(group ~ tte(time, censoring = event), data = df,
	  method.inference = "none", trace = 0, method.tte = "Gehan")
\end{lstlisting}
\begin{verbatim}
endpoint threshold  delta  Delta
    time     1e-12 0.3403 0.3403
\end{verbatim}

Expected:
\lstset{language=r,label= ,caption= ,captionpos=b,numbers=none}
\begin{lstlisting}
e.coxph <- coxph(Surv(time,event)~group,data = df)
HR <- as.double(exp(coef(e.coxph)))
c("HR" = alphaY/alphaX, "Delta" = 2*alphaX/(alphaY+alphaX)-1)
c("HR.cox" = HR, "Delta" = (1-HR)/(1+HR))
\end{lstlisting}

\begin{verbatim}
       HR     Delta 
0.5000000 0.3333333
   HR.cox     Delta 
0.4918256 0.3406392
\end{verbatim}

\clearpage

\section{Competing risks}
\label{sec:org5c403da}

\subsection{Theory}
\label{sec:orge858b85}

\subsubsection{General case (no censoring)}
\label{sec:orgb460310}
Let consider: 
\begin{itemize}
\item \(X^*_{E}\) the time to the occurrence of the event of interest in the control group.
\item \(Y^*_{E}\) the time to the occurrence of the event of interest in the treatment group.
\item \(X^*_{CR}\) the time to the occurrence of the competing event of interest in the control group.
\item \(Y^*_{CR}\) the time to the occurrence of the competing event of interest in the treatment group.
\end{itemize}
Let denote \(\varepsilon_X = 1 +\Ind[X^*_{E} > X^*_{CR}]\) the event type
indicator in the control group and \(\varepsilon_Y = 1 + \Ind[Y^*_{E} >
Y^*_{CR}]\) the event type indicator in treatment group (\(=1\) when the
cause of interest is realised first and 2 when the competing risk is
realised first).

\bigskip

For each subject either the event of interest or the competing event
is realized. We now define:
\begin{align*}
X = \left\{
              \begin{array}{ll}
                 X^*_{E} \text{ if }\varepsilon_X = 1  \\
                 +\infty \text{ if }\varepsilon_X = 2 
                \end{array}
              \right.
\text{ and }
Y = \left\{
              \begin{array}{ll}
                 Y^*_{E} \text{ if }\varepsilon_Y = 1  \\
                 +\infty \text{ if }\varepsilon_Y = 2 
                \end{array}
              \right.
\end{align*}
i.e. when the event of interest is not realized we say that the time to event is infinite.

\bigskip

We thus have:
\begin{align*}
\Prob[Y > X] 
= & \Prob[Y > X|\varepsilon_X=1,\varepsilon_Y=1]\Prob[\varepsilon_X=1,\varepsilon_Y=1] \\
&+ \Prob[Y > X|\varepsilon_X=1,\varepsilon_Y=2]\Prob[\varepsilon_X=1,\varepsilon_Y=2] \\
&+ \Prob[Y > X|\varepsilon_X=2,\varepsilon_Y=1]\Prob[\varepsilon_X=2,\varepsilon_Y=1] \\
&+ \Prob[Y > X|\varepsilon_X=2,\varepsilon_Y=2]\Prob[\varepsilon_X=2,\varepsilon_Y=2] \\
= & \Prob[Y > X|\varepsilon_X=1,\varepsilon_Y=1]\Prob[\varepsilon_X=1,\varepsilon_Y=1] \\
&+ 1*\Prob[\varepsilon_X=1,\varepsilon_Y=2] \\
&+ 0*\Prob[\varepsilon_X=2,\varepsilon_Y=1] \\
&+ 0*\Prob[\varepsilon_X=2,\varepsilon_Y=2] \\
\end{align*}

So \(\Prob[X > Y] = \Prob[X >
Y|\varepsilon_X=1,\varepsilon_Y=1]\Prob[\varepsilon_X=1,\varepsilon_Y=1] +
\Prob[\varepsilon_X=1,\varepsilon_Y=2]\) and:
\begin{align*}
\Delta = &
 \big(\Prob[X > Y|\varepsilon_X=1,\varepsilon_Y=1] - \Prob[X < Y|\varepsilon_X=1,\varepsilon_Y=1] \big) \Prob[\varepsilon_X=1,\varepsilon_Y=1] \\
& + \Prob[\varepsilon_X=1,\varepsilon_Y=2] - \Prob[\varepsilon_X=2,\varepsilon_Y=1]
\end{align*}

\subsubsection{General case (censoring, method: Gehan)}
\label{sec:org94c6c32}
In case of censoring we can use an inverse probability weighting
approach. Let denote \(\delta_{c,X}\) (resp. \(\delta_{c,Y}\)) the
indicator of no censoring relative to \(\tilde{X}\) (resp \(\tilde{Y}\)), \(\tilde{X}_E\) and \(\tilde{Y}_E\) the
censored event time. We can use inverse probability weighting to
compute the net benefit:
\begin{align*}
\Delta^{IPW} &= \frac{\delta_{c,\tilde{X}}\delta_{c,\tilde{Y}}}{\Prob[\delta_{c,\tilde{X}}]\Prob[\delta_{c,\tilde{Y}}]} (\Ind[\tilde{Y}>\tilde{X}]-\Ind[\tilde{Y}<\tilde{X}])\\
&= \left\{
                \begin{array}{ll}
                  \frac{1}{\Prob[\delta_{c,\tilde{X}}]\Prob[\delta_{c,\tilde{Y}}]} (\Ind[Y>X]-\Ind[Y<X])\text{, if no censoring}\\
                  0\text{, if censoring}
                \end{array}
              \right.
\end{align*}

This is equivalent to weight the informative pairs (i.e. favorable,
unfavorable and neutral) by the inverse of the complement of the
probability of being uninformative. This is what is done by the
argument \texttt{correction.tte} of \texttt{BuyseTest}. This works whenever the
censoring mechanism is independent of the event times and we have a
consistent estimate of \(\Prob[\delta_c]\) since:
\begin{align*}
\Esp[\Delta^{IPW}] &= \Esp\left[ \Esp\left[ \frac{\delta_{c,\tilde{X}}\delta_{c,\tilde{Y}}}{\Prob[\delta_{c,\tilde{X}}]\Prob[\delta_{c,\tilde{Y}}]} (\Ind[\tilde{Y}>\tilde{X}]-\Ind[\tilde{Y}<\tilde{X}]) \Bigg| \tilde{X}, \tilde{Y} \right] \right]\\
&= \Esp\left[\Esp\left[\frac{\delta_{c,\tilde{X}}\delta_{c,\tilde{Y}}}{\Prob[\delta_{c,\tilde{X}}]\Prob[\delta_{c,\tilde{Y}}]} \Bigg| \tilde{X}, \tilde{Y} \right]\right] \Esp\left[\Ind[Y>X]-\Ind[Y<X]\right]\\
&= \frac{\Esp\left[\delta_{c,\tilde{X}}\delta_{c,\tilde{Y}} \right]}{\Prob[\delta_{c,\tilde{X}}]\Prob[\delta_{c,\tilde{Y}}]} \Delta
= \frac{\Esp[\delta_{c,\tilde{X}}]\Esp[\delta_{c,\tilde{Y}}]}{\Prob[\delta_{c,\tilde{X}}]\Prob[\delta_{c,\tilde{Y}}]} \Delta\\
&= \Delta
\end{align*}
where we used the law of total expectation (first line) and the independence between the censoring mecanisms.

\subsubsection{Exponential distribution (no censoring)}
\label{sec:org691d9de}

Now let's assume that:
\begin{itemize}
\item \(X_{E} \sim Exp(\alpha_{E,X})\).
\item \(Y_{E} \sim Exp(\alpha_{E,Y})\).
\item \(X_{CR} \sim Exp(\alpha_{CR,X})\).
\item \(Y_{CR} \sim Exp(\alpha_{CR,Y})\).
\end{itemize}

Then:
\begin{align*}
 \Prob[Y_{E} > X_{E}] &= \Prob[Y_{E} >
X_{E}|\varepsilon_X=1,\varepsilon_Y=1]\Prob[\varepsilon_X=1,\varepsilon_Y=1] +
\Prob[\varepsilon_X=1,\varepsilon_Y=2] \\
&= \frac{1}{(\alpha_{E,X}+\alpha_{CR,X})(\alpha_{E,Y}+\alpha_{CR,Y})} \left(
 \alpha_{E,X}\alpha_{E,Y} \frac{\alpha_{E,X}}{\alpha_{E,X}+\alpha_{E,Y}}
+ \alpha_{E,X}\alpha_{CR,Y} \right) \\
\end{align*}


Just for comparison let's compare to the cumulative incidence. First
we only consider one group and two competing events whose times to
event follow an exponential distribution:
\begin{itemize}
\item \(T_E \sim Exp(\alpha_E)\). The corresponding hazard function is \(\lambda(t)=\alpha_E\).
\item \(T_{CR} \sim Exp(\alpha_{CR})\). The corresponding hazard function is \(\lambda(t)=\alpha_{CR}\).
\end{itemize}
The cumulative incidence function can be written:
\begin{align*}
CIF_1(t) &= \int_0^t \lambda_1(s) S(s_-) ds \\
&= \int_0^t \alpha_E \exp(- (\alpha_E + \alpha_{CR}) * s_-) ds \\
&= \frac{\alpha_E}{\alpha_E + \alpha_{CR}} \left[ \exp(- (\alpha_E + \alpha_{CR}) * s_-)\right]_t^0 \\
&= \frac{\alpha_E}{\alpha_E + \alpha_{CR}} \left(1 - \exp(- (\alpha_E + \alpha_{CR}) * t_-)\right) 
\end{align*}
where \(S(t)\) denote the event free survival and \(s_-\) denotes the right sided limit.

\bigskip

Then applying this formula in the case of two groups gives:
\begin{align*}
CIF_1(t|group = X) &= \frac{\alpha_{E,X}}{\alpha_{E,X} + \alpha_{CR,X}} \left(1 - \exp(- (\alpha_{E,X} + \alpha_{CR,X}) * t_-)\right) \\
CIF_1(t|group = Y) &= \frac{\alpha_{E,Y}}{\alpha_{E,Y} + \alpha_{CR,Y}} \left(1 - \exp(- (\alpha_{E,Y} + \alpha_{CR,Y}) * t_-)\right) 
\end{align*}

\subsection{In R}
\label{sec:org8180793}

\subsubsection{BuyseTest (no censoring)}
\label{sec:orgb33f6bb}

Setting:
\lstset{language=r,label= ,caption= ,captionpos=b,numbers=none}
\begin{lstlisting}
alphaE.X <- 2
alphaCR.X <- 1
alphaE.Y <- 3
alphaCR.Y <- 2
\end{lstlisting}

Simulate data:
\lstset{language=r,label= ,caption= ,captionpos=b,numbers=none}
\begin{lstlisting}
set.seed(10)
df <- rbind(data.frame(time1 = rexp(n, rate = alphaE.X), time2 = rexp(n, rate = alphaCR.X), group = "1"),
	    data.frame(time1 = rexp(n, rate = alphaE.Y), time2 = rexp(n, rate = alphaCR.Y), group = "2"))
df$time <- pmin(df$time1,df$time2) ## first event
df$event <- (df$time2<df$time1)+1 ## type of event
\end{lstlisting}

BuyseTest:
\lstset{language=r,label= ,caption= ,captionpos=b,numbers=none}
\begin{lstlisting}
e.BT <- BuyseTest(group ~ tte(time, censoring = event), data = df,
		  method.inference = "none", method.tte = "Gehan",
		  trace = 0)
summary(e.BT, percentage = TRUE)
\end{lstlisting}

\begin{verbatim}
       Generalized pairwise comparison with 1 prioritized endpoint

> statistic       : net chance of a better outcome (delta: endpoint specific, Delta: global) 
> null hypothesis : Delta == 0 
> treatment groups: 1 (control) vs. 2 (treatment) 
> censored pairs  : uninformative pairs

> results
endpoint threshold total favorable unfavorable neutral uninf   delta   Delta
    time     1e-12   100      41.6       45.12   13.28     0 -0.0352 -0.0352
\end{verbatim}

Note that without censoring one can get the same results by treating
time as a continuous variable that take value \(\infty\) when the
competing risk is observed:
\lstset{language=r,label= ,caption= ,captionpos=b,numbers=none}
\begin{lstlisting}
df$timeXX <- df$time
df$timeXX[df$event==2] <- max(df$time)+1
e.BT.bis <- BuyseTest(group ~ cont(timeXX), data = df,
		  method.inference = "none", trace = 0)
summary(e.BT.bis, percentage = TRUE)
\end{lstlisting}

\begin{verbatim}
       Generalized pairwise comparison with 1 prioritized endpoint

> statistic       : net chance of a better outcome (delta: endpoint specific, Delta: global) 
> null hypothesis : Delta == 0 
> treatment groups: 1 (control) vs. 2 (treatment) 
> results
endpoint threshold total favorable unfavorable neutral uninf   delta   Delta
  timeXX     1e-12   100      41.6       45.12   13.28     0 -0.0352 -0.0352
\end{verbatim}

Expected:
\lstset{language=r,label= ,caption= ,captionpos=b,numbers=none}
\begin{lstlisting}
weight <- (alphaE.X+alphaCR.X)*(alphaE.Y+alphaCR.Y)
exp <- list()
exp$favorable <- 1/weight*(alphaE.X*alphaE.Y*alphaE.X/(alphaE.X+alphaE.Y)+(alphaE.X*alphaCR.Y))
exp$unfavorable <- 1/weight*(alphaE.X*alphaE.Y*alphaE.Y/(alphaE.X+alphaE.Y)+(alphaE.Y*alphaCR.X))
exp$neutral <- alphaCR.X*alphaCR.Y/weight

100*unlist(exp)
\end{lstlisting}

\begin{verbatim}
favorable unfavorable     neutral 
 42.66667    44.00000    13.33333
\end{verbatim}

\subsubsection{BuyseTest (with censoring)}
\label{sec:org568d439}

Simulate data:
\lstset{language=r,label= ,caption= ,captionpos=b,numbers=none}
\begin{lstlisting}
df$eventC <- df$event
df$eventC[rbinom(n, size = 1, prob = 0.2)==1] <- 0
\end{lstlisting}

BuyseTest (biased):
\lstset{language=r,label= ,caption= ,captionpos=b,numbers=none}
\begin{lstlisting}
e.BTC <- BuyseTest(group ~ tte(time, censoring = eventC), data = df,
		   method.inference = "none", method.tte = "Gehan",
		   trace = 0)
summary(e.BTC, percentage = TRUE)
\end{lstlisting}

\begin{verbatim}
       Generalized pairwise comparison with 1 prioritized endpoint

> statistic       : net chance of a better outcome (delta: endpoint specific, Delta: global) 
> null hypothesis : Delta == 0 
> treatment groups: 1 (control) vs. 2 (treatment) 
> censored pairs  : uninformative pairs

> results
endpoint threshold total favorable unfavorable neutral uninf   delta   Delta
    time     1e-12   100      31.1       35.15    8.65  25.1 -0.0406 -0.0406
\end{verbatim}

BuyseTest (unbiased):
\lstset{language=r,label= ,caption= ,captionpos=b,numbers=none}
\begin{lstlisting}
e.BTCC <- BuyseTest(group ~ tte(time, censoring = eventC), data = df,
		   method.inference = "none", method.tte = "Gehan corrected",
		   trace = 0)
summary(e.BTCC, percentage = TRUE)
\end{lstlisting}

\begin{verbatim}
       Generalized pairwise comparison with 1 prioritized endpoint

> statistic       : net chance of a better outcome (delta: endpoint specific, Delta: global) 
> null hypothesis : Delta == 0 
> treatment groups: 1 (control) vs. 2 (treatment) 
> censored pairs  : uninformative pairs
                    IPW for uninformative pairs

> results
endpoint threshold total favorable unfavorable neutral uninf   delta   Delta
    time     1e-12   100     41.52       46.94   11.54     0 -0.0542 -0.0542
\end{verbatim}

\subsubsection{Cumulative incidence}
\label{sec:org7bf5324}

Settings:
\lstset{language=r,label= ,caption= ,captionpos=b,numbers=none}
\begin{lstlisting}
alphaE <- 2
alphaCR <- 1
\end{lstlisting}

Simulate data:
\lstset{language=r,label= ,caption= ,captionpos=b,numbers=none}
\begin{lstlisting}
set.seed(10)
df <- data.frame(time1 = rexp(n, rate = alphaE), time2 = rexp(n, rate = alphaCR), group = "1", event = 1)
df$time <- pmin(df$time1,df$time2)
df$event <- (df$time2<df$time1)+1
\end{lstlisting}

Cumulative incidence (via risk regression):
\lstset{language=r,label= ,caption= ,captionpos=b,numbers=none}
\begin{lstlisting}
e.CSC <- CSC(Hist(time, event) ~ 1, data = df)
vec.times <- unique(round(exp(seq(log(min(df$time)),log(max(df$time)),length.out = 12)),2))
e.CSCpred <- predict(e.CSC, newdata = data.frame(X = 1), time = vec.times , cause = 1)
\end{lstlisting}

Expected vs. calculated:
\lstset{language=r,label= ,caption= ,captionpos=b,numbers=none}
\begin{lstlisting}
cbind(time = vec.times,
      CSC = e.CSCpred$absRisk[1,],
      manual = alphaE/(alphaE+alphaCR)*(1-exp(-(alphaE+alphaCR)*(vec.times)))
      )
\end{lstlisting}

\begin{verbatim}
     time    CSC     manual
[1,] 0.00 0.0000 0.00000000
[2,] 0.01 0.0186 0.01970298
[3,] 0.02 0.0377 0.03882364
[4,] 0.05 0.0924 0.09286135
[5,] 0.14 0.2248 0.22863545
[6,] 0.42 0.4690 0.47756398
[7,] 1.24 0.6534 0.65051069
[8,] 3.70 0.6703 0.66665659
\end{verbatim}

Could also be obtained treating the outcome as binary:
\lstset{language=r,label= ,caption= ,captionpos=b,numbers=none}
\begin{lstlisting}
mean((df$time<=1)*(df$event==1))
\end{lstlisting}

\begin{verbatim}
[1] 0.6375
\end{verbatim}
\end{document}