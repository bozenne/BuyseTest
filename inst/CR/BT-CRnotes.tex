% Created 2018-07-10 ti 18:22
% Intended LaTeX compiler: pdflatex
\documentclass{article}

%%%% settings when exporting code %%%% 

\usepackage{listings}
\lstset{
backgroundcolor=\color{white},
basewidth={0.5em,0.4em},
basicstyle=\ttfamily\small,
breakatwhitespace=false,
breaklines=true,
columns=fullflexible,
commentstyle=\color[rgb]{0.5,0,0.5},
frame=single,
keepspaces=true,
keywordstyle=\color{black},
literate={~}{$\sim$}{1},
numbers=left,
numbersep=10pt,
numberstyle=\ttfamily\tiny\color{gray},
showspaces=false,
showstringspaces=false,
stepnumber=1,
stringstyle=\color[rgb]{0,.5,0},
tabsize=4,
xleftmargin=.23in,
emph={anova,apply,class,coef,colnames,colNames,colSums,dim,dcast,for,ggplot,head,if,ifelse,is.na,lapply,list.files,library,logLik,melt,plot,require,rowSums,sapply,setcolorder,setkey,str,summary,tapply},
emphstyle=\color{blue}
}

%%%% packages %%%%%

\usepackage[utf8]{inputenc}
\usepackage[T1]{fontenc}
\usepackage{lmodern}
\usepackage{textcomp}
\usepackage{color}
\usepackage{enumerate}
\usepackage{graphicx}
\usepackage{grffile}
\usepackage{wrapfig}
\usepackage{capt-of}
\usepackage{caption}
\usepackage{rotating}
\usepackage{longtable}
\usepackage{multirow}
\usepackage{multicol}
\usepackage{changes}
\usepackage{pdflscape}
\usepackage{geometry}
\usepackage[normalem]{ulem}
\usepackage{amssymb}
\usepackage{amsmath}
\usepackage{amsfonts}
\usepackage{dsfont}
\usepackage{textcomp}
\usepackage{array}
\usepackage{ifthen}
\usepackage{hyperref}
\usepackage{natbib}
\RequirePackage{fancyvrb}
\DefineVerbatimEnvironment{verbatim}{Verbatim}{fontsize=\small,formatcom = {\color[rgb]{0.5,0,0}}}
\RequirePackage{colortbl} % arrayrulecolor to mix colors
\RequirePackage{setspace} % to modify the space between lines - incompatible with footnote in beamer
\usepackage{authblk} % enable several affiliations (clash with beamer)
\RequirePackage{epstopdf} % to be able to convert .eps to .pdf image files
%
%%%% additional latex commands %%%%
%
\RequirePackage{amsmath}
\RequirePackage{algorithm}
\RequirePackage[noend]{algpseudocode}
\RequirePackage{ifthen}
\RequirePackage{xspace} % space for newcommand macro
\RequirePackage{xifthen}
\RequirePackage{xargs}
\RequirePackage{dsfont}
\RequirePackage{amsmath,stmaryrd,graphicx}
\RequirePackage{prodint} % product integral symbol (\PRODI)
\RequirePackage{amsthm}
\newtheorem{theorem}{Theorem}
\newtheorem{lemma}[theorem]{Lemma}
\newcommand\defOperator[7]{%
\ifthenelse{\isempty{#2}}{
\ifthenelse{\isempty{#1}}{#7{#3}#4}{#7{#3}#4 \left#5 #1 \right#6}
}{
\ifthenelse{\isempty{#1}}{#7{#3}#4_{#2}}{#7{#3}#4_{#1}\left#5 #2 \right#6}
}
}
\newcommand\defUOperator[5]{%
\ifthenelse{\isempty{#1}}{
#5\left#3 #2 \right#4
}{
\ifthenelse{\isempty{#2}}{\underset{#1}{\operatornamewithlimits{#5}}}{
\underset{#1}{\operatornamewithlimits{#5}}\left#3 #2 \right#4}
}
}
\newcommand{\defBoldVar}[2]{
\ifthenelse{\equal{#2}{T}}{\boldsymbol{#1}}{\mathbf{#1}}
}
\newcommandx\Cov[2][1=,2=]{\defOperator{#1}{#2}{C}{ov}{\lbrack}{\rbrack}{\mathbb}}
\newcommandx\Esp[2][1=,2=]{\defOperator{#1}{#2}{E}{}{\lbrack}{\rbrack}{\mathbb}}
\newcommandx\Prob[2][1=,2=]{\defOperator{#1}{#2}{P}{}{\lbrack}{\rbrack}{\mathbb}}
\newcommandx\Qrob[2][1=,2=]{\defOperator{#1}{#2}{Q}{}{\lbrack}{\rbrack}{\mathbb}}
\newcommandx\Var[2][1=,2=]{\defOperator{#1}{#2}{V}{ar}{\lbrack}{\rbrack}{\mathbb}}
\newcommandx\Binom[2][1=,2=]{\defOperator{#1}{#2}{B}{}{(}{)}{\mathcal}}
\newcommandx\Gaus[2][1=,2=]{\defOperator{#1}{#2}{N}{}{(}{)}{\mathcal}}
\newcommandx\Wishart[2][1=,2=]{\defOperator{#1}{#2}{W}{ishart}{(}{)}{\mathcal}}
\newcommandx\Likelihood[2][1=,2=]{\defOperator{#1}{#2}{L}{}{(}{)}{\mathcal}}
\newcommandx\Information[2][1=,2=]{\defOperator{#1}{#2}{I}{}{(}{)}{\mathcal}}
\newcommandx\Score[2][1=,2=]{\defOperator{#1}{#2}{S}{}{(}{)}{\mathcal}}
\newcommandx\Vois[2][1=,2=]{\defOperator{#1}{#2}{V}{}{(}{)}{\mathcal}}
\newcommandx\IF[2][1=,2=]{\defOperator{#1}{#2}{IF}{}{(}{)}{\mathcal}}
\newcommandx\Ind[1][1=]{\defOperator{}{#1}{1}{}{(}{)}{\mathds}}
\newcommandx\Max[2][1=,2=]{\defUOperator{#1}{#2}{(}{)}{min}}
\newcommandx\Min[2][1=,2=]{\defUOperator{#1}{#2}{(}{)}{max}}
\newcommandx\argMax[2][1=,2=]{\defUOperator{#1}{#2}{(}{)}{argmax}}
\newcommandx\argMin[2][1=,2=]{\defUOperator{#1}{#2}{(}{)}{argmin}}
\newcommandx\cvD[2][1=D,2=n \rightarrow \infty]{\xrightarrow[#2]{#1}}
\newcommandx\Hypothesis[2][1=,2=]{
\ifthenelse{\isempty{#1}}{
\mathcal{H}
}{
\ifthenelse{\isempty{#2}}{
\mathcal{H}_{#1}
}{
\mathcal{H}^{(#2)}_{#1}
}
}
}
\newcommandx\dpartial[4][1=,2=,3=,4=\partial]{
\ifthenelse{\isempty{#3}}{
\frac{#4 #1}{#4 #2}
}{
\left.\frac{#4 #1}{#4 #2}\right\rvert_{#3}
}
}
\newcommandx\dTpartial[3][1=,2=,3=]{\dpartial[#1][#2][#3][d]}
\newcommandx\ddpartial[3][1=,2=,3=]{
\ifthenelse{\isempty{#3}}{
\frac{\partial^{2} #1}{\left( \partial #2\right)^2}
}{
\frac{\partial^2 #1}{\partial #2\partial #3}
}
}
\newcommand\Real{\mathbb{R}}
\newcommand\Rational{\mathbb{Q}}
\newcommand\Natural{\mathbb{N}}
\newcommand\trans[1]{{#1}^\intercal}%\newcommand\trans[1]{{\vphantom{#1}}^\top{#1}}
\newcommand{\independent}{\mathrel{\text{\scalebox{1.5}{$\perp\mkern-10mu\perp$}}}}
\newcommand\half{\frac{1}{2}}
\newcommand\normMax[1]{\left|\left|#1\right|\right|_{max}}
\newcommand\normTwo[1]{\left|\left|#1\right|\right|_{2}}
\author{Brice Ozenne, Julien Péron}
\date{\today}
\title{Explicit formula for the net benefit}
\hypersetup{
 colorlinks=true,
 citecolor=[rgb]{0,0.5,0},
 urlcolor=[rgb]{0,0,0.5},
 linkcolor=[rgb]{0,0,0.5},
 pdfauthor={Brice Ozenne, Julien Péron},
 pdftitle={Explicit formula for the net benefit},
 pdfkeywords={},
 pdfsubject={},
 pdfcreator={Emacs 25.2.1 (Org mode 9.0.4)},
 pdflang={English}
 }
\begin{document}

\maketitle
\tableofcontents

\clearpage

\section{Parameter of interest}
\label{sec:org3ea6593}

Let consider two independent random variables \(X\) and \(Y\).
We are interested in:
\begin{align*}
\Delta = \Prob[Y>X] - \Prob[X>Y]
\end{align*}

\bigskip

In the examples we will use a sample size of:
\lstset{language=r,label= ,caption= ,captionpos=b,numbers=none}
\begin{lstlisting}
n <- 1e4
\end{lstlisting}
and use the following R packages
\lstset{language=r,label= ,caption= ,captionpos=b,numbers=none}
\begin{lstlisting}
library(BuyseTest)
library(riskRegression)
\end{lstlisting}

\clearpage

\section{Binary variable}
\label{sec:orgf6303b5}

\subsection{Theory}
\label{sec:org0f0fda7}
\begin{align*}
\Prob[Y>X] = \Prob[Y=1,X=0]
\end{align*}
Using the independence between \(Y\) and \(X\):
\begin{align*}
\Prob[Y>X] = \Prob[Y=1]\Prob[X=0] = \Prob[Y=1](1-\Prob[X=0]) = \Prob[Y=1] - \Prob[Y=1]\Prob[X=1]
\end{align*}
By symmetry:
\begin{align*}
\Prob[X>Y] = \Prob[X=1] - \Prob[Y=1]\Prob[X=1]
\end{align*}
So 
\begin{align*}
\Delta = \Prob[Y=1] - \Prob[X=0]
\end{align*}

\subsection{In R}
\label{sec:orgaa5aa4b}
Settings:
\lstset{language=r,label= ,caption= ,captionpos=b,numbers=none}
\begin{lstlisting}
prob1 <- 0.4
prob2 <- 0.2
\end{lstlisting}

Simulate data:
\lstset{language=r,label= ,caption= ,captionpos=b,numbers=none}
\begin{lstlisting}
set.seed(10)
df <- rbind(data.frame(tox = rbinom(n, prob = prob1, size = 1), group = "C"),
	    data.frame(tox = rbinom(n, prob = prob2, size = 1), group = "T"))
\end{lstlisting}

Buyse test:
\lstset{language=r,label= ,caption= ,captionpos=b,numbers=none}
\begin{lstlisting}
BuyseTest(group ~ bin(tox), data = df, method.inference = "none", trace = 0)
\end{lstlisting}
\begin{verbatim}
endpoint threshold   delta   Delta
     tox       0.5 -0.1981 -0.1981
\end{verbatim}

Expected:
\lstset{language=r,label= ,caption= ,captionpos=b,numbers=none}
\begin{lstlisting}
prob2 - prob1
\end{lstlisting}

\begin{verbatim}
[1] -0.2
\end{verbatim}

\clearpage

\section{Continuous variable}
\label{sec:orgb9cd910}

\subsection{Theory}
\label{sec:org5628198}
Let's consider two normally distributed variables with common variance:
\begin{itemize}
\item \(X \sim \Gaus[\mu_X,\sigma^2]\)
\item \(Y \sim \Gaus[\mu_Y,\sigma^2]\)
\end{itemize}
Denoting \(d = \frac{\mu_Y-\mu_X}{\sigma}\): 
\begin{itemize}
\item \(X^* \sim \Gaus[0,1]\)
\item \(Y^* \sim \Gaus[d,1]\)
\end{itemize}
\begin{align*}
\Prob[Y>X] &= \Esp[\Ind[Y>X]] = \Esp[\Ind[Y*>X*]] = \Esp[\Ind[Z>0]]
\end{align*}
where \(Z \sim \Gaus[d,2]\) so \(\Prob[Y>X] = \Phi(\frac{d}{\sqrt{2}})\)

By symmetry
\begin{align*}
\Delta = 2*\Phi(\frac{d}{\sqrt{2}})-1
\end{align*}

\subsection{In R}
\label{sec:org259d92b}

Settings:
\lstset{language=r,label= ,caption= ,captionpos=b,numbers=none}
\begin{lstlisting}
mean1 <- 0
mean2 <- 2
sd12 <- 1
\end{lstlisting}

Simulate data:
\lstset{language=r,label= ,caption= ,captionpos=b,numbers=none}
\begin{lstlisting}
set.seed(10)
df <- rbind(data.frame(tox = rnorm(n, mean = mean1, sd = sd12), group = "C"),
	    data.frame(tox = rnorm(n, mean = mean2, sd = sd12), group = "T"))
\end{lstlisting}

Buyse test:
\lstset{language=r,label= ,caption= ,captionpos=b,numbers=none}
\begin{lstlisting}
BuyseTest(group ~ cont(tox), data = df, method.inference = "none", trace = 0)
\end{lstlisting}

\begin{verbatim}
endpoint threshold  delta  Delta
     tox     1e-12 0.8359 0.8359
\end{verbatim}

Expected:
\lstset{language=r,label= ,caption= ,captionpos=b,numbers=none}
\begin{lstlisting}
d <- (mean2-mean1)/sd12
2*pnorm(d/sqrt(2))-1
\end{lstlisting}

\begin{verbatim}
[1] 0.8427008
\end{verbatim}

\clearpage

\section{Survival}
\label{sec:org5ee3ff8}

\subsection{Theory}
\label{sec:org1c2a052}
For a given cumulative density function \(F(x)\) and a corresponding
probability density function \(f(x)\) we define the hazard by:
\begin{align*}
\lambda(t) &=  \left. \frac{\Prob[t\leq T \leq t+h|T\geq t]}{h}\right|_{h \rightarrow 0^+} \\
&= \left. \frac{\Prob[t\leq T \leq t+h]}{\Prob[T\geq t]h}\right|_{h \rightarrow 0^+} \\
&= \frac{f(t)}{1-F(t)}
\end{align*}

\bigskip

Let now consider two times to events following an exponential distribution:
\begin{itemize}
\item \(T1 \sim Exp(\alpha_1)\). The corresponding hazard function is \(\lambda(t)=\alpha_1\).
\item \(T2 \sim Exp(\alpha_2)\). The corresponding hazard function is \(\lambda(t)=\alpha_2\).
\end{itemize}
So the hazad ratio is \(HR = \frac{\lambda_1}{\lambda_2}\). Note that if we use a cox model we will have:
\begin{align*}
\lambda(t) = \lambda_0(t) \exp(\beta \Ind[group])
\end{align*}
where \(\exp(\beta)\) is the hazard ratio.

\bigskip

\begin{align*}
\Prob[T_1>T_2] &= \int_{0}^{\infty}\alpha_1 \exp(-\alpha_1 t)  \int_0^{t_1} \alpha_2 \exp(-\alpha_2 t) dt_2 dt_1 \\
&= \int_{0}^{\infty}\alpha_1 \exp(-\alpha_1 t)  [ \exp(-\alpha_2 t) ]_{t_1}^{0} dt_1 \\
&= \int_{0}^{\infty}\alpha_1 \exp(-\alpha_1 t)  ( \exp(-\alpha_2 t_1) - 1 ) dt_1 \\
&= \frac{\alpha_1}{\alpha_1+\alpha_2} [\exp(-(\alpha_1+\alpha_2) t)]_{\infty}^{0} - [\exp(-\alpha_1 t)]_{\infty}^{0} \\
&= 1-\frac{\alpha_1}{\alpha_1+\alpha_2}\\
&= 1-\frac{HR}{1+HR}\\
\end{align*}
So \(\Prob[T_2>T_1]=\frac{HR}{1+HR}\).

\subsection{In R}
\label{sec:org9da9a72}

Settings:
\lstset{language=r,label= ,caption= ,captionpos=b,numbers=none}
\begin{lstlisting}
alpha1 <- 2
alpha2 <- 1
\end{lstlisting}

Simulate data:
\lstset{language=r,label= ,caption= ,captionpos=b,numbers=none}
\begin{lstlisting}
set.seed(10)
df <- rbind(data.frame(time = rexp(n, rate = alpha1), group = "C", event = 1),
	    data.frame(time = rexp(n, rate = alpha2), group = "T", event = 1))
\end{lstlisting}

Buyse test:
\lstset{language=r,label= ,caption= ,captionpos=b,numbers=none}
\begin{lstlisting}
BuyseTest(group ~ tte(time, censoring = event), data = df,
	  method.inference = "none", trace = 0, method.tte = "Gehan")
\end{lstlisting}
\begin{verbatim}
endpoint threshold  delta  Delta
    time     1e-12 0.3403 0.3403
\end{verbatim}

Expected:
\lstset{language=r,label= ,caption= ,captionpos=b,numbers=none}
\begin{lstlisting}
e.coxph <- coxph(Surv(time,event)~group,data = df)
HR <- as.double(exp(coef(e.coxph)))
c("HR" = alpha2/alpha1, "Delta" = (alpha2/alpha1)/(1+alpha2/alpha1))
c("HR.cox" = HR, "Delta" = (HR)/(1+HR))
\end{lstlisting}

\begin{verbatim}
       HR     Delta 
0.5000000 0.3333333
   HR.cox     Delta 
0.4918256 0.3296804
\end{verbatim}

\clearpage

\section{Competing risks}
\label{sec:orga8d6f18}

\subsection{Theory}
\label{sec:orgd30e7b6}

Let now consider two competing events whose times to event follow an exponential distribution:
\begin{itemize}
\item \(T1 \sim Exp(\alpha_1)\). The corresponding hazard function is \(\lambda(t)=\alpha_1\).
\item \(T2 \sim Exp(\alpha_2)\). The corresponding hazard function is \(\lambda(t)=\alpha_2\).
\end{itemize}
The cumulative incidence function can be written:
\begin{align*}
CIF_1(t) &= \int_0^t \lambda_1(s) S(s_-) ds \\
&= \int_0^t \alpha_1 \exp(- (\alpha_1 + \alpha_2) * s_-) ds \\
&= \frac{\alpha_1}{\alpha_1 + \alpha_2} \left[ \exp(- (\alpha_1 + \alpha_2) * s_-)\right]_t^0 \\
&= \frac{\alpha_1}{\alpha_1 + \alpha_2} \left(1 - \exp(- (\alpha_1 + \alpha_2) * t_-)\right) 
\end{align*}
where \(S(t)\) denote the event free survival and \(s_-\) denotes the right sided limit.

\bigskip

Now if we consider two groups such that:
\begin{itemize}
\item \(T1 \sim Exp(\alpha_{1,T})\) in group \(T\) and \(T1 \sim Exp(\alpha_{1,C})\) in group \(C\)
\item \(T2 \sim Exp(\alpha_{2,T})\) in group \(T\) and \(T2 \sim Exp(\alpha_{2,C})\) in group \(C\)
\end{itemize}

Then:
\begin{align*}
CIF_1(t|group = T) &= \frac{\alpha_{1,T}}{\alpha_{1,T} + \alpha_{2,T}} \left(1 - \exp(- (\alpha_{1,T} + \alpha_{2,T}) * t_-)\right) \\
CIF_1(t|group = C) &= \frac{\alpha_{1,C}}{\alpha_{1,C} + \alpha_{2,C}} \left(1 - \exp(- (\alpha_{1,C} + \alpha_{2,C}) * t_-)\right) 
\end{align*}

\bigskip

Let denote \(\varepsilon_T\) the event type indicator (1 cause of
interest and 2 competing risk) in group \(T\) and \(\varepsilon_C\)
the event type indicator in group \(C\):
\begin{align*}
\Delta &= \frac{1}{\Prob[\varepsilon_T=1,\varepsilon_C=1]} \frac{\alpha_{1,T}}{\alpha_{1,T}+\alpha_{1,C}}
- \frac{1}{\Prob[\varepsilon_T=1,\varepsilon_C=2]}
+ \frac{1}{\Prob[\varepsilon_T=2,\varepsilon_C=1]} \\
&=
\end{align*}

\subsection{In R}
\label{sec:org3bacfa5}

Settings:
\lstset{language=r,label= ,caption= ,captionpos=b,numbers=none}
\begin{lstlisting}
alpha1 <- 2
alpha2 <- 1
\end{lstlisting}

Simulate data:
\lstset{language=r,label= ,caption= ,captionpos=b,numbers=none}
\begin{lstlisting}
set.seed(10)
df <- data.frame(time1 = rexp(n, rate = alpha1), time2 = rexp(n, rate = alpha2), group = "1", event = 1)
df$time <- pmin(df$time1,df$time2)
df$event <- (df$time2<df$time1)+1
\end{lstlisting}

Cumulative incidence (via risk regression):
\lstset{language=r,label= ,caption= ,captionpos=b,numbers=none}
\begin{lstlisting}
e.CSC <- CSC(Hist(time, event) ~ 1, data = df)
vec.times <- unique(round(exp(seq(log(min(df$time)),log(max(df$time)),length.out = 12)),2))
e.CSCpred <- predict(e.CSC, newdata = data.frame(X = 1), time = vec.times , cause = 1)
\end{lstlisting}

Expected vs. calculated:
\lstset{language=r,label= ,caption= ,captionpos=b,numbers=none}
\begin{lstlisting}
cbind(time = vec.times,
      CSC = e.CSCpred$absRisk[1,],
      manual = alpha1/(alpha1+alpha2)*(1-exp(-(alpha1+alpha2)*(vec.times)))
      )
\end{lstlisting}

\begin{verbatim}
     time    CSC     manual
[1,] 0.00 0.0000 0.00000000
[2,] 0.01 0.0186 0.01970298
[3,] 0.02 0.0377 0.03882364
[4,] 0.05 0.0924 0.09286135
[5,] 0.14 0.2248 0.22863545
[6,] 0.42 0.4690 0.47756398
[7,] 1.24 0.6534 0.65051069
[8,] 3.70 0.6703 0.66665659
\end{verbatim}

Could also be obtained treating the outcome as binary:
\lstset{language=r,label= ,caption= ,captionpos=b,numbers=none}
\begin{lstlisting}
mean((df$time<=1)*(df$event==1))
\end{lstlisting}

\begin{verbatim}
[1] 0.6375
\end{verbatim}

Now with Buyse test:
\lstset{language=r,label= ,caption= ,captionpos=b,numbers=none}
\begin{lstlisting}
df11 <- df[]
\end{lstlisting}
\end{document}