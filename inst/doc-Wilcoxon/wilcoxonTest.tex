% Created 2024-06-14 fr 14:02
% Intended LaTeX compiler: pdflatex
\documentclass[12pt]{article}

%%%% settings when exporting code %%%% 

\usepackage{listings}
\lstdefinestyle{code-small}{
backgroundcolor=\color{white}, % background color for the code block
basicstyle=\ttfamily\small, % font used to display the code
commentstyle=\color[rgb]{0.5,0,0.5}, % color used to display comments in the code
keywordstyle=\color{black}, % color used to highlight certain words in the code
numberstyle=\ttfamily\tiny\color{gray}, % color used to display the line numbers
rulecolor=\color{black}, % color of the frame
stringstyle=\color[rgb]{0,.5,0},  % color used to display strings in the code
breakatwhitespace=false, % sets if automatic breaks should only happen at whitespace
breaklines=true, % sets automatic line breaking
columns=fullflexible,
frame=single, % adds a frame around the code (non,leftline,topline,bottomline,lines,single,shadowbox)
keepspaces=true, % % keeps spaces in text, useful for keeping indentation of code
literate={~}{$\sim$}{1}, % symbol properly display via latex
numbers=none, % where to put the line-numbers; possible values are (none, left, right)
numbersep=10pt, % how far the line-numbers are from the code
showspaces=false,
showstringspaces=false,
stepnumber=1, % the step between two line-numbers. If it's 1, each line will be numbered
tabsize=1,
xleftmargin=0cm,
emph={anova,apply,class,coef,colnames,colNames,colSums,dim,dcast,for,ggplot,head,if,ifelse,is.na,lapply,list.files,library,logLik,melt,plot,require,rowSums,sapply,setcolorder,setkey,str,summary,tapply},
aboveskip = \medskipamount, % define the space above displayed listings.
belowskip = \medskipamount, % define the space above displayed listings.
lineskip = 0pt} % specifies additional space between lines in listings
\lstset{style=code-small}
%%%% packages %%%%%

\usepackage[utf8]{inputenc}
\usepackage[T1]{fontenc}
\usepackage{lmodern}
\usepackage{textcomp}
\usepackage{color}
\usepackage{graphicx}
\usepackage{grffile}
\usepackage{wrapfig}
\usepackage{rotating}
\usepackage{longtable}
\usepackage{multirow}
\usepackage{multicol}
\usepackage{changes}
\usepackage{pdflscape}
\usepackage{geometry}
\usepackage[normalem]{ulem}
\usepackage{amssymb}
\usepackage{amsmath}
\usepackage{amsfonts}
\usepackage{dsfont}
\usepackage{array}
\usepackage{ifthen}
\usepackage{hyperref}
\usepackage{natbib}
\RequirePackage{setspace} % to modify the space between lines - incompatible with footnote in beamer
\renewcommand{\baselinestretch}{1.1}
\geometry{a4paper, left=10mm, right=10mm, top=10mm}
\usepackage{titlesec}
\usepackage{etoolbox}

\makeatletter
\patchcmd{\ttlh@hang}{\parindent\z@}{\parindent\z@\leavevmode}{}{}
\patchcmd{\ttlh@hang}{\noindent}{}{}{}
\makeatother
\RequirePackage{colortbl} % arrayrulecolor to mix colors
\definecolor{myorange}{rgb}{1,0.2,0}
\definecolor{mypurple}{rgb}{0.7,0,8}
\definecolor{mycyan}{rgb}{0,0.6,0.6}
\newcommand{\lightblue}{blue!50!white}
\newcommand{\darkblue}{blue!80!black}
\newcommand{\darkgreen}{green!50!black}
\newcommand{\darkred}{red!50!black}
\definecolor{gray}{gray}{0.5}
\hypersetup{
citecolor=[rgb]{0,0.5,0},
urlcolor=[rgb]{0,0,0.5},
linkcolor=[rgb]{0,0,0.5},
}
\newenvironment{note}{\small \color{gray}\fontfamily{lmtt}\selectfont}{\par}
\newenvironment{activity}{\color{orange}\fontfamily{qzc}\selectfont}{\par}
\RequirePackage{pifont}
\RequirePackage{relsize}
\newcommand{\Cross}{{\raisebox{-0.5ex}%
{\relsize{1.5}\ding{56}}}\hspace{1pt} }
\newcommand{\Valid}{{\raisebox{-0.5ex}%
{\relsize{1.5}\ding{52}}}\hspace{1pt} }
\newcommand{\CrossR}{ \textcolor{red}{\Cross} }
\newcommand{\ValidV}{ \textcolor{green}{\Valid} }
\usepackage{stackengine}
\usepackage{scalerel}
\newcommand\Warning[1][3ex]{%
\renewcommand\stacktype{L}%
\scaleto{\stackon[1.3pt]{\color{red}$\triangle$}{\tiny\bfseries !}}{#1}%
\xspace
}
\newcommand\Rlogo{\textbf{\textsf{R}}\xspace} %
\RequirePackage{fancyvrb}
\DefineVerbatimEnvironment{verbatim}{Verbatim}{fontsize=\small,formatcom = {\color[rgb]{0.5,0,0}}}
\RequirePackage{enumitem} % better than enumerate
\RequirePackage{epstopdf} % to be able to convert .eps to .pdf image files
\RequirePackage{capt-of} %
\RequirePackage{caption} % newlines in graphics
\RequirePackage{tikz-cd} % graph
\RequirePackage{booktabs} % for nice lines in table (e.g. toprule, bottomrule, midrule, cmidrule)
\RequirePackage{amsmath}
\RequirePackage{algorithm}
\RequirePackage[noend]{algpseudocode}
\RequirePackage{dsfont}
\RequirePackage{amsmath,stmaryrd,graphicx}
\RequirePackage{prodint} % product integral symbol (\PRODI)
\usepackage{ifthen}
\usepackage{xifthen}
\usepackage{xargs}
\usepackage{xspace}
\newcommand\defOperator[7]{%
\ifthenelse{\isempty{#2}}{
\ifthenelse{\isempty{#1}}{#7{#3}#4}{#7{#3}#4 \left#5 #1 \right#6}
}{
\ifthenelse{\isempty{#1}}{#7{#3}#4_{#2}}{#7{#3}#4_{#1}\left#5 #2 \right#6}
}
}
\newcommand\defUOperator[5]{%
\ifthenelse{\isempty{#1}}{
#5\left#3 #2 \right#4
}{
\ifthenelse{\isempty{#2}}{\underset{#1}{\operatornamewithlimits{#5}}}{
\underset{#1}{\operatornamewithlimits{#5}}\left#3 #2 \right#4}
}
}
\newcommand{\defBoldVar}[2]{
\ifthenelse{\equal{#2}{T}}{\boldsymbol{#1}}{\mathbf{#1}}
}
\newcommandx\Esp[2][1=,2=]{\defOperator{#1}{#2}{E}{}{\lbrack}{\rbrack}{\mathbb}}
\newcommandx\Prob[2][1=,2=]{\defOperator{#1}{#2}{P}{}{\lbrack}{\rbrack}{\mathbb}}
\newcommandx\Qrob[2][1=,2=]{\defOperator{#1}{#2}{Q}{}{\lbrack}{\rbrack}{\mathbb}}
\newcommandx\Var[2][1=,2=]{\defOperator{#1}{#2}{V}{ar}{\lbrack}{\rbrack}{\mathbb}}
\newcommandx\Cov[2][1=,2=]{\defOperator{#1}{#2}{C}{ov}{\lbrack}{\rbrack}{\mathbb}}
\newcommandx\Binom[2][1=,2=]{\defOperator{#1}{#2}{B}{}{(}{)}{\mathcal}}
\newcommandx\Gaus[2][1=,2=]{\defOperator{#1}{#2}{N}{}{(}{)}{\mathcal}}
\newcommandx\Wishart[2][1=,2=]{\defOperator{#1}{#2}{W}{ishart}{(}{)}{\mathcal}}
\newcommandx\Likelihood[2][1=,2=]{\defOperator{#1}{#2}{L}{}{(}{)}{\mathcal}}
\newcommandx\logLikelihood[2][1=,2=]{\defOperator{#1}{#2}{\ell}{}{(}{)}{}}
\newcommandx\Information[2][1=,2=]{\defOperator{#1}{#2}{I}{}{(}{)}{\mathcal}}
\newcommandx\Score[2][1=,2=]{\defOperator{#1}{#2}{S}{}{(}{)}{\mathcal}}
\newcommandx\Vois[2][1=,2=]{\defOperator{#1}{#2}{V}{}{(}{)}{\mathcal}}
\newcommandx\IF[2][1=,2=]{\defOperator{#1}{#2}{IF}{}{(}{)}{\mathcal}}
\newcommandx\Ind[1][1=]{\defOperator{}{#1}{1}{}{(}{)}{\mathds}}
\newcommandx\Max[2][1=,2=]{\defUOperator{#1}{#2}{(}{)}{min}}
\newcommandx\Min[2][1=,2=]{\defUOperator{#1}{#2}{(}{)}{max}}
\newcommandx\argMax[2][1=,2=]{\defUOperator{#1}{#2}{(}{)}{argmax}}
\newcommandx\argMin[2][1=,2=]{\defUOperator{#1}{#2}{(}{)}{argmin}}
\newcommandx\cvD[2][1=D,2=n \rightarrow \infty]{\xrightarrow[#2]{#1}}
\newcommandx\Hypothesis[2][1=,2=]{
\ifthenelse{\isempty{#1}}{
\mathcal{H}
}{
\ifthenelse{\isempty{#2}}{
\mathcal{H}_{#1}
}{
\mathcal{H}^{(#2)}_{#1}
}
}
}
\newcommandx\dpartial[4][1=,2=,3=,4=\partial]{
\ifthenelse{\isempty{#3}}{
\frac{#4 #1}{#4 #2}
}{
\left.\frac{#4 #1}{#4 #2}\right\rvert_{#3}
}
}
\newcommandx\dTpartial[3][1=,2=,3=]{\dpartial[#1][#2][#3][d]}
\newcommandx\ddpartial[3][1=,2=,3=]{
\ifthenelse{\isempty{#3}}{
\frac{\partial^{2} #1}{\partial #2^2}
}{
\frac{\partial^2 #1}{\partial #2\partial #3}
}
}
\newcommand\Real{\mathbb{R}}
\newcommand\Rational{\mathbb{Q}}
\newcommand\Natural{\mathbb{N}}
\newcommand\trans[1]{{#1}^\intercal}%\newcommand\trans[1]{{\vphantom{#1}}^\top{#1}}
\newcommand{\independent}{\mathrel{\text{\scalebox{1.5}{$\perp\mkern-10mu\perp$}}}}
\newcommand\half{\frac{1}{2}}
\newcommand\normMax[1]{\left|\left|#1\right|\right|_{max}}
\newcommand\normTwo[1]{\left|\left|#1\right|\right|_{2}}
\newcommand\Veta{\boldsymbol{\eta}}
\newcommand\VX{\mathbf{X}}
\author{Brice Ozenne}
\date{\today}
\title{Wilcoxon test via GPC}
\hypersetup{
 colorlinks=true,
 pdfauthor={Brice Ozenne},
 pdftitle={Wilcoxon test via GPC},
 pdfkeywords={},
 pdfsubject={},
 pdfcreator={Emacs 27.2 (Org mode 9.5.2)},
 pdflang={English}
 }
\begin{document}

\maketitle
Generalized Pairwise comparisons (GPC) include the Wilcoxon rank sum
test as a specific case. This vignette explores the connections
between the \texttt{BuyseTest} output and the more standard implementation of
the wilcoxon-test.


\section{Single Wilcoxon test}
\label{sec:org7e791f9}

\subsection{Exact test}
\label{sec:org9cd8bae}

Consider the first 'two sample test' example from the help page section of \texttt{stats::wilcox.test}:
\lstset{language=r,label= ,caption= ,captionpos=b,numbers=none}
\begin{lstlisting}
x <- c(0.80, 0.83, 1.89, 1.04, 1.45, 1.38, 1.91, 1.64, 0.73, 1.46)
y <- c(1.15, 0.88, 0.90, 0.74, 1.21)
df <- rbind(data.frame(value = x, group="x"),
            data.frame(value = y, group="y"))
\end{lstlisting}

\noindent We can perform a Wilcoxon test using the \texttt{wilcox.test} function:

\lstset{language=r,label= ,caption= ,captionpos=b,numbers=none}
\begin{lstlisting}
wilcox.test(value ~ group, data = df)
\end{lstlisting}

\begin{verbatim}

	Wilcoxon rank sum exact test

data:  value by group
W = 35, p-value = 0.2544
alternative hypothesis: true location shift is not equal to 0
\end{verbatim}


which, with such a small sample, can perform an exact test, i.e.,
consider all possible permutation of the group variable. It
unfortunately does not ouput any effect size (just the test statistic
and corresponding p-value). The package \emph{asht} contains an alternative
implementation:
\lstset{language=r,label= ,caption= ,captionpos=b,numbers=none}
\begin{lstlisting}
asht::wmwTest(value ~ group, data = df, method = "exact.ce")
\end{lstlisting}

\begin{verbatim}

	exact Wilcoxon-Man-Whitney test (confidence interval requires proportional odds
	assumption, but test does not)

data:  value by group
Mann-Whitney estimate = 0.3, p-value = 0.2544
alternative hypothesis: two distributions are not equal
95 percent confidence interval:
 0.08292978 0.63269022
sample estimates:
Mann-Whitney estimate 
                  0.3
\end{verbatim}

\noindent which output an estimate, the probability that a randomly
chosen observation from one group has higher value than a randomly
chosen observation from the other group, which is refered to as
Mann-Whitney parameter or probabilistic index. To match those results
with GPC we can use a permutation test: \newline \Warning the argument
\texttt{add.halfNeutral} should be set to \texttt{TRUE} to adequatly handle ties
\lstset{language=r,label= ,caption= ,captionpos=b,numbers=none}
\begin{lstlisting}
eperm.BT <- BuyseTest(group ~ cont(value), data = df, add.halfNeutral = TRUE,
                      method.inference = "permutation", n.resampling = 1e4,
                      trace = FALSE, cpus = 5, seed = 10)
confint(eperm.BT, statistic = "favorable")
\end{lstlisting}

\begin{verbatim}
      estimate        se lower.ci upper.ci null   p.value
value      0.3 0.1632342       NA       NA  0.5 0.2566743
\end{verbatim}


Up to the Monte Carlo error for the p-value calculation, which can be
made arbitrarily small by increasing the number of permutations
(argument \texttt{n.resampling}), the resuls are identical. Note that the
'default' statistical inference method based on asymptotic
U-statistic theory:
\lstset{language=r,label= ,caption= ,captionpos=b,numbers=none}
\begin{lstlisting}
eU.BT <- BuyseTest(group ~ cont(value), data = df,
                  add.halfNeutral = TRUE)
confint(eU.BT, statistic = "favorable")
\end{lstlisting}

\begin{verbatim}
      estimate        se  lower.ci  upper.ci null  p.value
value      0.3 0.1334166 0.1098282 0.5981833  0.5 0.182315
\end{verbatim}


\noindent leads to a different p-value as a different null hypothesis
is being tested here: probabilistic index equal 0.5 instead of
equality in distribution. This p-value corresponds, up to some small
sample approximation and the Monte Carlo error, to the one obtain with
a studentized permutation:
\lstset{language=r,label= ,caption= ,captionpos=b,numbers=none}
\begin{lstlisting}
etperm.BT <- BuyseTest(group ~ cont(value), data = df, add.halfNeutral = TRUE,
                      method.inference = "studentized permutation", n.resampling = 1e4,
                      trace = FALSE, seed = 10)
confint(etperm.BT, statistic = "favorable")
\end{lstlisting}

\begin{verbatim}
      estimate        se lower.ci upper.ci null   p.value
value      0.3 0.1334166       NA       NA  0.5 0.1916808
\end{verbatim}

\subsection{Approximate test}
\label{sec:org54056ae}

Consider now a bigger (artificial) dataset:
\lstset{language=r,label= ,caption= ,captionpos=b,numbers=none}
\begin{lstlisting}
set.seed(10)
df2 <- rbind(data.frame(value = round(rnorm(50),2), group="x"),
             data.frame(value = round(rnorm(50),2), group="y"))
any(duplicated(df2$value)) ## test whether there are any ties
\end{lstlisting}

\begin{verbatim}
[1] TRUE
\end{verbatim}


We can again perform a Wilcoxon test using the \texttt{wilcox.test} function:
\lstset{language=r,label= ,caption= ,captionpos=b,numbers=none}
\begin{lstlisting}
wilcox.test(value ~ group, data = df2)
\end{lstlisting}

\begin{verbatim}

	Wilcoxon rank sum test with continuity correction

data:  value by group
W = 967.5, p-value = 0.05188
alternative hypothesis: true location shift is not equal to 0
\end{verbatim}


or, equivalenty, with the \texttt{wmwTest} function:
\lstset{language=r,label= ,caption= ,captionpos=b,numbers=none}
\begin{lstlisting}
wmwTest(value ~ group, data = df2)
\end{lstlisting}

\begin{verbatim}

	Wilcoxon-Mann-Whitney test with continuity correction (confidence
	interval requires proportional odds assumption, but test does not)

data:  value by group
Mann-Whitney estimate = 0.613, tie factor = 0.99995, p-value = 0.05188
alternative hypothesis: two distributions are not equal
95 percent confidence interval:
 0.4990803 0.7138973
sample estimates:
Mann-Whitney estimate 
                0.613
\end{verbatim}

In either case, an exact test would be too computationally demanding
and an approximate test is performed instead, which assumes a normaly
distributed test statistic. The BuyseTest package will not be able to
match these results due to the continuity correction. Without
continuity correction, e.g.:
\lstset{language=r,label= ,caption= ,captionpos=b,numbers=none}
\begin{lstlisting}
wmwTest(value ~ group, data = df2, correct = FALSE)
\end{lstlisting}

\begin{verbatim}

	Wilcoxon-Mann-Whitney test (confidence interval requires proportional odds
	assumption, but test does not)

data:  value by group
Mann-Whitney estimate = 0.613, tie factor = 0.99995, p-value = 0.05147
alternative hypothesis: two distributions are not equal
95 percent confidence interval:
 0.4992803 0.7137196
sample estimates:
Mann-Whitney estimate 
                0.613
\end{verbatim}

\noindent it is possible to retrieve the exact same p-value by
evaluating the variance of the permutation distribution and assuming a
normally distributed test statistic. In this simple example this can
be done using an analytic formula \citep{anderson2023exact}: \newline
\Warning the code, kindly provided by the authors of the paper, has
been ported to the package with minimal change. It is therefore meant
to be used in the context of the original publication and not in the
more general setting covered by the package (strata, right-censoring,
\ldots)
\lstset{language=r,label= ,caption= ,captionpos=b,numbers=none}
\begin{lstlisting}
eperm.BT2 <- BuyseTest(group ~ cont(value), data = df2, add.halfNeutral = TRUE,
                       method.inference = "varexact-permutation")
confint(eperm.BT2, statistic = "favorable")
\end{lstlisting}

\begin{verbatim}
      estimate         se lower.ci upper.ci null    p.value
value    0.613 0.05802219       NA       NA  0.5 0.05147115
\end{verbatim}


or, more generally, using a resampling method:
\lstset{language=r,label= ,caption= ,captionpos=b,numbers=none}
\begin{lstlisting}
eperm.BT2 <- BuyseTest(group ~ cont(value), data = df2, add.halfNeutral = TRUE,
                       method.inference = "permutation", n.resampling = 1e4,
                       trace = FALSE, cpus = 5, seed = 10)
confint(eperm.BT2, statistic = "favorable", method.ci.resampling = "gaussian")
\end{lstlisting}

\begin{verbatim}
      estimate         se lower.ci upper.ci null    p.value
value    0.613 0.05814569       NA       NA  0.5 0.05118099
\end{verbatim}


\clearpage

\section{Multiple Wilcoxon tests}
\label{sec:org8342996}

Consider now the case where we would like to compare one reference
group (here strata \texttt{a}) to multiple treatment groups (here strata
\texttt{b,c,d,e}). We will consider the following dataset:
\lstset{language=r,label= ,caption= ,captionpos=b,numbers=none}
\begin{lstlisting}
set.seed(35)
dt <- simBuyseTest(n.T=25, n.strata = 5)
dt$id <- paste0("id",1:NROW(dt))
dt$strata <- as.character(dt$strata) 
head(dt)
\end{lstlisting}

\begin{verbatim}
       id treatment  eventtime status toxicity      score strata
   <char>    <fctr>      <num>  <num>   <fctr>      <num> <char>
1:    id1         C 0.03384999      1      yes  0.4777913      b
2:    id2         C 0.65039474      0       no -1.1048190      d
3:    id3         C 1.00647502      1       no -0.1407630      b
4:    id4         C 0.01129603      1      yes -0.5512507      a
5:    id5         C 0.22249748      1       no  1.0465250      d
6:    id6         C 0.07400412      0       no -2.0053855      d
\end{verbatim}


We can apply the GPC procedure to each pair of group:
\lstset{language=r,label= ,caption= ,captionpos=b,numbers=none}
\begin{lstlisting}
BuyseTest.options(order.Hprojection=1);BuyseTest.options(trace=0)

ls.BT <- list("b-a=0" = BuyseTest(strata ~ cont(score), add.halfNeutral = TRUE,
                                  data = dt[dt$strata %in% c("a","b"),],
                                  method.inference = "u-statistic"),
              "c-a=0" = BuyseTest(strata ~ cont(score), add.halfNeutral = TRUE,
                                  data = dt[dt$strata %in% c("a","c"),],
                                  method.inference = "u-statistic"),
              "d-a=0" = BuyseTest(strata ~ cont(score), add.halfNeutral = TRUE,
                                  data = dt[dt$strata %in% c("a","d"),],
                                  method.inference = "u-statistic"),
              "e-a=0" = BuyseTest(strata ~ cont(score), add.halfNeutral = TRUE,
                                  data = dt[dt$strata %in% c("a","e"),],
                                  method.inference = "u-statistic")
              )

M.confint <- do.call(rbind,lapply(ls.BT,confint, statistic = "favorable"))
cbind(M.confint,adj.p.value = p.adjust(M.confint[,"p.value"], method = "bonferroni"))
\end{lstlisting}

\begin{verbatim}
       estimate        se  lower.ci  upper.ci null    p.value adj.p.value
b-a=0 0.4090909 0.1542200 0.1654639 0.7073759  0.5 0.56434599   1.0000000
c-a=0 0.4375000 0.1465755 0.1948678 0.7142379  0.5 0.67306460   1.0000000
d-a=0 0.2500000 0.1010153 0.1039078 0.4893302  0.5 0.04143057   0.1657223
e-a=0 0.3333333 0.1360828 0.1308601 0.6241219  0.5 0.25767454   1.0000000
\end{verbatim}



Because we compare the treatment groups to the same reference, the
test statistics are correlated and a Bonferroni adjustment would not
be optimal. A better (but still not optimal adjustment) is the
max-test adjustment which can be obtained via the \texttt{BuyseMultComp} function:
\lstset{language=r,label= ,caption= ,captionpos=b,numbers=none}
\begin{lstlisting}
e.mc <- BuyseMultComp(ls.BT, statistic = "favorable", cluster = "id", global = TRUE)
print(e.mc, cols = c("estimate","se","p.value","adj.p.value"))
\end{lstlisting}

\begin{verbatim}
  - Multivariate test: p.value = 0.2645493 (df = 4)
  - Univariate tests:
       estimate        se    p.value adj.p.value
b-a=0 0.4090909 0.1542200 0.56434599   0.9289219
c-a=0 0.4375000 0.1465755 0.67306460   0.9752151
d-a=0 0.2500000 0.1010153 0.04143057   0.1223430
e-a=0 0.3333333 0.1360828 0.25767454   0.5831344
\end{verbatim}



Here the smallest p-value has been multiplied by a factor 2.64 instead
of 4. This is thanks to the rather strong correlation between the test
statistics:
\lstset{language=r,label= ,caption= ,captionpos=b,numbers=none}
\begin{lstlisting}
M.cor <- cor(lava::iid(e.mc))
dimnames(M.cor) <- list(names(ls.BT),names(ls.BT))
M.cor
\end{lstlisting}

\begin{verbatim}
          b-a=0     c-a=0     d-a=0     e-a=0
b-a=0 1.0000000 0.6519486 0.5601058 0.7520401
c-a=0 0.6519486 1.0000000 0.4240003 0.5439927
d-a=0 0.5601058 0.4240003 1.0000000 0.5051815
e-a=0 0.7520401 0.5439927 0.5051815 1.0000000
\end{verbatim}


\section*{References}
\label{sec:org5adc982}
\begingroup
\renewcommand{\section}[2]{}

\bibliographystyle{apalike}
\bibliography{bibliography}

\endgroup
\end{document}